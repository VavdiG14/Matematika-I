\documentclass[11pt]{article}
\usepackage[utf8]{inputenc}
\usepackage[slovene]{babel}
\usepackage{amsthm,amsfonts,amsmath,amssymb,url}

\textheight 210 true mm
\textwidth 146 true mm
\voffset=-17mm
\hoffset=-13mm

\newtheorem{Izrek}{{\sc Izrek}}[section]
\newtheorem{Trditev}[Izrek]{{\sc Trditev}}
\newtheorem{Posledica}[Izrek]{{\sc Posledica}}
\newtheorem{Definicija}[Izrek]{{\sc Definicija}}
\newtheorem{Zgled}[Izrek]{{\sc Zgled}}
\newtheorem{Opomba}[Izrek]{{\sc Opomba}}
\def\theIzrek{{\rm \arabic{section}.\arabic{Izrek}}}

\newenvironment{izrek}{\begin{Izrek}\sl}{\end{Izrek}}
\newenvironment{trditev}{\begin{Trditev}\sl}{\end{Trditev}}
\newenvironment{posledica}{\begin{Posledica}\sl}{\end{Posledica}}
\newenvironment{definicija}{\begin{Definicija}\rm }{\end{Definicija}}
\newenvironment{zgled}{\begin{Zgled}\rm }{\end{Zgled}}
\newenvironment{opomba}{\begin{Opomba}\rm }{\end{Opomba}}

\newenvironment{dokaz}[1][{\sc Dokaz}]{\begin{proof}[#1]\renewcommand*{\qedsymbol}{\(\blacksquare\)}}{\end{proof}}

\newcommand{\Mod}[1]{\hbox{ (mod } #1)}

\begin{document}
	
	\thispagestyle{empty}
	\begin{center}
		\begin{Large}
			{\bf Zapiski pri predmetu Matematika 1}
		\end{Large}
		
	\end{center}
	V zapiskih so zbrani osnovni izreki, definicije in nekaj dokazov, ki smo jih delali na predavanjih. Za učenje še vedno priporočam, da imate ob sebi zvezek s predavanj, ker vseh stvari ni spisanih.
	Zapiske je potrebno razumeti kot minimalni katalog znanja.
	Za napake ne odgovarjam. Posebna zahavala Blažu Poljancu, za pomoč pri nastajanju zapiskov.
	\vfill
	\begin{center}
		Ljubljana, 2016 $\quad \quad $ Gregor Vavdi
	\end{center}
	\newpage
	\setcounter{page}{1}

%%%%%%%%%%%%%%%%%%%%%%%%%%%%%%%%%%%%%%%%%%%5TEVILA%%%%%%%%%%%%%%%%%%%%%%%%%%%%%%%%%%5
\section{Števila}
\subsection{Naravna števila}


\begin{Definicija}
	Z naravnimi števili štejemo. Na množici naravnih števil
	$$N = {1, 2, 3, . . .}$$
	sta naravno definirani računski operaciji:
	\newline
	$+$ seštevanje,
	\newline
	$\cdot$ množenje.
	\newline
	Pravimo, da je množica naravnih števil zaprta za seštevanje in množenje, saj sta vsota $a+b$ in podukt $a\cdot b$ poljubnih naravnih števil a in b tudi naravni števili. Naravnih števil ne moremo poljubno odštevati, saj npr. $5 - 7$ ni naravno število.
 \newline
	Množico naravnih števil je  \textbf{linearno urejena}:
	če $m,n\in \mathbb{N}$,potem velja bodisi $m\le n$ ali $n \le m$.
	\newline
	in je \textbf{dobro urejena}: 
	ima vsaka njena neprazna množica najmanjši element.
\end{Definicija}
\begin{Izrek}
	(Načelo popolne indukcije): Vsaka podmnožica $A \subset \mathbb{N}$, ki vsebuje število 1 in v kateri je poleg števila n vselej tudi njegov naslednik, vsebuje vsa naravna števila :
	$$(A \subset \mathbb{N}) \wedge (1 \in A) \wedge (n\in A \Rightarrow n^+ \in A) \Rightarrow A = \mathbb{N}$$
\end{Izrek}
\begin{Definicija}
	$$m,n \in \mathbb{N}: m < n \iff \exists k \in \mathbb{N}: n = m + k$$
	$$m,n \in \mathbb{N}: m \le n \iff m < n \lor m = n$$
\end{Definicija}

\subsection{Cela števila}
\begin{Definicija}
	Računske operacije z množico naravnih števil $\mathbb{N}$ na množico \textbf{celih števil} $\mathbb{Z}$. Na množici celih števil:
	$$\mathbb{Z} = \{\cdots, -2, -1,0 ,1,2,\cdots\}$$
	definiramo tri računske operacije:
	\newline
	$+$ seštevanje
	\newline
	$\cdot$ množenje
	\newline
	$-$ odštevanje
	\newline
	Pravimo, da je množica celih števil zaprta za seštevanje odštevanje in množenje. 
	\newline
	Množica celih števil je \textbf{linearno urejena}: poljubni celi števili lahko primerjamo po velikosti
	\newline
	\textbf{ni dobro urejena}: ker nima najmanjšega elementa.
	
\end{Definicija}

\subsection{Racionalna števila}

\begin{Definicija}
	Racionalna števila so ulomki oblike:
	$$\frac{m}{n} \quad m\in \mathbb{Z},\quad n\in\mathbb{N}$$
	Množica $\mathbb{Q}$ je komutativni obseg. (sešetvanje in množenje).
	\newline
	Velja:
	$$\frac{m}{n} < \frac{k}{l} \iff \exists \frac{i}{j}\quad i,j\in\mathbb{N}\Rightarrow \frac{m}{n} + \frac{i}{j} = \frac{k}{l}$$
\end{Definicija}
\begin{Trditev}
	Decimalno število, ki ima periodičen decimalni zapis je racionalno število.
\end{Trditev}
\begin{Opomba}
	Velja tudi obratno, vsako racionalno število ima (končen ali) periodičen decimalni zapis.
\end{Opomba}

\subsection{Realna števila}
\begin{Definicija}
	Realna števila so vsa števila, racionalna in iracionalna števila. Množico realnih števil označimo s $\mathbb{R}$. Realna števila so komutativni obseg.
	\\
	Realna števila imajo decimalni zapis. Če je decimalni zapis končen ali periodičen je število racionalno, sicer je število iracionalno.
	\\
	Vsakemu realnemu številu ustreza točka na številski premici in vsaki točki na številski premici ustreza realno število.
	\\
	Množica realnih števil je linearno urejena.
\end{Definicija}
\begin{Trditev}
	Vsak odprt interval vsebuje racionalno število
\end{Trditev}
\begin{Opomba}
	Racionalna števila so gosta v realnih številih.
\end{Opomba}
\begin{Trditev}
	Na vsakem intervalu leži iracionalno število.
\end{Trditev}
\begin{Definicija}
	$$\sqrt[m]{a}$$ je tisto pozitivno realno število, za katerega velja $ a = b^m$. Število b imenujemo \textbf{m-ti koren od a}.
\end{Definicija}
\subsection{Absolutna vrednost}
\begin{Definicija}
	Naj bo $x\in \mathbb{R}$. \textbf{Absolutna vrednost} $|x|$ je:
	\begin{displaymath}
	|x| = \left\{ \begin{array}{cc}
	 x & \textrm{$x\ge 0$},\\
	-x & \textrm{$x <0$}.
	\end{array} \right.
	\end{displaymath}
\end{Definicija}
\begin{Trditev}
	Lastnosti absolutne vrednosti:
	\begin{enumerate}
		\item
		$|x| \ge 0 \quad \forall x\in\mathbb{R}$
		\item
		$|x| = 0 \iff x = 0$
		\item
		$|-x| = |x|$
		\item
		$ -|x| \le x\le |x|$
		\item
		Geometrijski pomen $|x-y|$ - razdalja med x in y.
		\item
		$|x\cdot y| = |x| \cdot |y| \quad x,y\in \mathbb{R}$
		\item
		$|x+y| \le |x| +|y| \quad x,y\in \mathbb{R}$ - Trikotniška neenakost
		\item
		$||x| -|y|| \le |x\pm y| \le |x| + |y| \quad x,y\in \mathbb{R}$
		\item
		$|x_1 + x-2 +x_3 + \dots + x_n| \le |x_1| +|x_2| +|x_3| + \dots + |x_n| \quad \forall x_1,x_2,\dots, x_n$
		\item
		$|x| \le c \iff -c \le x \le c \quad c>0$
		\item
		$|x| \ge c \iff x \ge c$ ali $x \le -c$
	\end{enumerate}
\end{Trditev}
\subsection{Kompleksna števila}
\begin{Definicija}
	Množica kompleksnih števil je množica oblike \\ $z = a + ib \quad a,b\in \mathbb{R}$
	$$\mathbb{C} = \{a + ib: a,b\in \mathbb{R}\}$$
\end{Definicija}
\begin{Definicija}
	Naj bo $ z = a +bi \in \mathbb{C}$. 
	\\
	\textbf{Realni del} števila z je realno število a: $\Re(z) = a$
	\textbf{ Imaginarni del} števila z je realno število b: $\Im(z) = b$.
\end{Definicija}
\begin{Trditev}
	\begin{enumerate}
		\item
		Konjugacija kompleksnega števila:
		\\ 
		Naj bo $ z = a +bi \in \mathbb{C}$. Potem je konjugirano število $\bar{z} = a - bi$
		\item
		Absolutna vrednost kompleksnega števila
		\\
		Naj bo $z = a+bi \in \mathbb{C}$. Absolutna vrednost kompleksnega števila je nenegativno realno število: $|z| = \sqrt{a^2 + b^2}$.
		\item
		Enakost kompleksnih števil.
		\\
		Naj bosta $z, w$ kompleksni števili. $z =w$ natanko tedaj ko velja: $\Re(z) = \Re(w)$ in $\Im(z) =\Im(z)$.
		\item
		Seštevanje kompleksnih števil
		\\
		Naj bosta $z = a+ bi$ in $w = c+ di$ kompleksni števili. $$z + w = (a+c) + (b+d)i$$
		\item
		Množenje kompleksnih števil
		\\
		Naj bosta $z = a+bi$ in $ w = c + di$ kompleskni števili.
		$$ z\cdot w = (ac - bd) + (ad + bc)i$$
	\end{enumerate}
\end{Trditev}
\begin{Opomba}
	$\mathbb{C}$ ni linearno urejena. Zato zapis $z < w$ nima smisla.
\end{Opomba}
\begin{Trditev}
	Lastnosti konugacije:
	\\
	Naj bosta $z,w \in \mathbb{C}$
	\begin{enumerate}
		\item
		$\bar{\bar{z}} = z$
		\item
		$ z =\bar{z} \iff z \in \mathbb{R}$
		\item
		$ z + \bar{z} = 2\Re(z)$
		\item
		$\bar{z+w} = \bar{z} + \bar{w}$
		\item
		$\bar{z\cdot w} = \bar{z}\cdot \bar{w}$
	\end{enumerate}
\end{Trditev}
\begin{Trditev}
	Lastnosti absolutne vrednosti:
	\begin{enumerate}
		\item
		$|z| \ge 0$
		\item
		$|\Re(z)| \le |z|$
		\item
		$|\Im(z)| \le |z|$
		\item
		$|z\cdot w| = |z| \cdot |w|$
		\item
		$|z+w| \le |z| +|w|$
	\end{enumerate}
\end{Trditev}

\begin{Izrek}
	(Osnovni izrek algebre) \\
	Naj bo $p(z) = a_nz^n + a_{n-1}z^{n-1} + \dots + a_1z^1 + a_0 \quad n\in\mathbb{N}, n\ge 1, a_n \ne 0$. Potem obstaja kompleksno število  $z_0 \in\mathbb{C} : p(z_0 ) = 0$.
\end{Izrek}
\begin{Posledica}
	Vsak polinom stopnje $n\ge 1$ ima n kompleksnih ničel štetih z večkratnostjo.
\end{Posledica}
\begin{Trditev}
	Naj bo $p(x) = a_nx^n + a_{n-1}x^{n-1} + \dots + a_1x^1 + a_0$ polinom z realnimi koeficienti: $a_o,a_1, \dots, a_n \in \mathbb{R}$. Če je $p(z) = 0 $, za nek $z\in \mathbb{C}$, potem $p(\bar{z}) = 0$.
\end{Trditev}
\begin{Posledica}
	Vsak polinom z realnimi koeficienti, tj. $p(x) = a_nx^n + a_{n-1}x^{n-1} + \dots + a_1x^1 + a_0$ lahko v realnem rezcepimo do linearnih in kvadratnih nerazcepnih faktorjev.
\end{Posledica}
\section{Množice}
\begin{Definicija}
	Množica je zbirka objektov, ki jih imenujemo elementi množice. Če x priprada množici A, pišemo $x\in A$ in rečemo x je element množice A.
	Če x ne pripada množici A, pišemo: $\quad x\notin A$.
\end{Definicija}
\begin{Definicija}
	Prazna množica je množica, ki nima nobenega elementa. \\
	Pišemo: $\emptyset = \{\}$
	\\
	Množici A in B sta enaki, kadar vsebujeta iste elemente: $A=B$. Množici sta enaki, če in samo če velja:
	\\
	Za vsak $x\in A$, potem velja $x\in B$ in
	\\
	Za vsak $x\in B$, velja $x\in A$.
	\\
	Pravimo, da je A podmnožica množice B, če za vsak $x\in A $ velja $x\in B$, pišemo $A\subset B$.
\end{Definicija}
\begin{Definicija}
	\textbf{Unija} množice A in B je množica, ki vsebuje vse elemente množice A in vse elemente množice B. 
	$$A \cup B = \{x: x\in A \lor x\in B\}$$
\end{Definicija}
\begin{Definicija}
	\textbf{Presek} množice A in B je množica, ki je sestavljena iz elementov x, ki pripadajo A in B.
	$$A\cap U = \{x: x\in A \land  x \in B\}$$
\end{Definicija}
\begin{Opomba}
	Pravimo, da sta množici \textbf{disjunktni}, če je $A\cap B = \emptyset$.
\end{Opomba}
\begin{Definicija}
	Če najprej predpišemo množico, tako da so vse množice s katerimi računamo njene podmnožice, to množico imenujemo \textbf{univerzalna množica } $\mathcal{U}$.
\end{Definicija}
\begin{Definicija}
	Naj bo $A \subset \mathcal{U}$. \textbf{Komplement množice} A je množica vseh elementov $x\in \mathcal{U}$, ki ne pripadajo A.
	$$A^C = \{x\in\mathcal{U}: x\notin A \}$$
\end{Definicija}
\begin{Definicija}
	Naj bosta A in B množici. \textbf{Razlika} množice A in B je množica, ki vsebuje vse tiste elemente množice A, ki ne ležijo v B.
	$$A\backslash B = \{x: x\in A \land x\notin B \}$$
	Lastnost: $A^C = \mathcal{U}\backslash A$.
\end{Definicija}
\begin{Definicija}
	\textbf{Kartezični produkt} množice A in B je množica urejenih parov $(a,b)$, kjer je $a\in A$ in $b\in B$.
	$$A \times B = \{(a,b): a\in A \land b\in B \}$$
\end{Definicija}
\begin{Definicija}
	Naj bo S podmnožica $\mathbb{R}$. Če obstaja tak $M\in\mathbb{R}$,da je $x\le M$ za vsak $x\in S$, pravimo, da je \textbf{M zgornja meja} množice S.
	\\
	Če ima S kakšno zgornjo mejo, pravimo, da je S \textbf{navzgor omejena}. Če je $M$ zgornja meja množice S, je tudi vsako število, ki je večje od M, zgornja meja množice S.
	\\
	Pomembna je najmanjša zgornja meja, ki se imenuje natančna zgornja meja ali \textbf{supremum} množice S in jo označimo s $\sup S$.
\end{Definicija}
\begin{Definicija}
	(Dedekindov aksiom) Vsaka neprazna, navzgor omejena množica realnih števil ima neko realno število za natančno zgornjo mejo.
\end{Definicija}
\begin{Opomba}
	Dedakindov aksiom ne velja za racionalna števila.
\end{Opomba}
\begin{Opomba}
	Naj bo S neprazna množica navzgor omejena podmnožica v $\mathbb{R}$. $\alpha = \sup S$ natanko tedaj, ko veljata:
	\begin{enumerate}
		\item
		$\alpha$ je zgornja meja množice S, to je:
		$$ s\le \alpha \quad \forall s\in S$$
		\item
		$\alpha$ je med vsemi zgornjimi mejami množice S najmanjša, to je:
		$$\forall \beta < \alpha \quad \exists s\in S \Rightarrow s >\beta$$
	\end{enumerate}
\end{Opomba}
\begin{Definicija}
	Naj bo S podmnožica $\mathbb{R}$. Če obstaja tak $m\in\mathbb{R}$,da je $x\ge m$ za vsak $x\in S$, pravimo, da je \textbf{M spodnja meja} množice S.
	\\
	Če ima S kakšno spodnjo mejo, pravimo, da je S \textbf{navzdol omejena}. Največja spodnja meja,če obstaja, imenujemo \textbf{infimum} množice S in jo označimo z $\inf S$.
\end{Definicija}
\section{Številska zaporedja}

\begin{Definicija}
	Funkciji $f: \mathbb{N}\to \mathbb{R}$ rečemo \textbf{realno zaporedje}. Označimo $a_n = f(n)$ in $a_n$ imenujemo $n$-ti člen zaporedja. Zaporedje lahko podamo z $a_1, a_2, a_3,...$ ali pa na kratko $\{a_n\}_{n\in \mathbb{N}}$.
\end{Definicija}

\begin{Definicija}
	Naj bo $\{a_n\}_{n\in \mathbb{N}}$ realno zaporedje:
	\\
	\indent če velja $a_n \le a_{n+1},\forall n \in \mathbb{N}$, pravimo, da je zaporedje \textbf{naraščajoče}.
	\\
	\indent	če velja $a_n < a_{n+1},\forall n \in \mathbb{N}$, pravimo, da je zaporedje \textbf{strogo naraščajoče}.
	\\
	\indent če velja $a_n \ge a_{n+1},\forall n \in \mathbb{N}$, pravimo, da je zaporedje \textbf{padajoče}.
	\\
	\indent če velja $a_n > a_{n+1},\forall n \in \mathbb{N}$, pravimo, da je zaporedje \textbf{strogo padajoče}.
\end{Definicija}

\begin{Definicija}
	Naj bo $\{a_n\}_{n\in \mathbb{N}}$ realno zaporedje. Zaporedje je \textbf{monotono}, če bodisi naraščajoče, bodisi padajoče.
	Zaporedje je \textbf{strogo monotono}, če je bodisi strogo naraščajoče, bodisi strogo padajoče.
\end{Definicija}

\begin{Definicija}
	Zgornja meja zaporedja $\{a_n\}_{n\in \mathbb{N}}$ je zgornja meja množice $\{a_1, a_2,a_3,...\}$. Če obstaja kakšna zgornja meja, potem pravimo, da je zaporedje \textbf{navzgor omejeno}. V tem primeru najmanjšo zgornjo mejo imenujemo \textbf{supremum}.
\end{Definicija}

\begin{Definicija}
		Naj bo $\{a_n\}_{n\in \mathbb{N}}$ zaporedje realnih števil. \textbf{Stekališče} zaporedja  $\{a_n\}_{n\in \mathbb{N}}$ je tako število $s \in \mathbb{R}$, da je za vsak $\varepsilon > 0$, velja $| s - a_n | < \varepsilon$, za neskončno mnogo indeksov $n\in\mathbb{N}$.
\end{Definicija}
\begin{Opomba}
	Število $s$ je stekališče zaporedja $\{a_n\}_{n\in \mathbb{N}}$, natanko tedaj, kadar v njegovi $\varepsilon$- okolici leži neskončno mnogo členov zaporedja.
\end{Opomba}

\begin{Izrek}
	Vsako omejeno zaporedje ima stekališče.
\end{Izrek}


\begin{dokaz}
	
	$\exists m, M : m \le a_n \le M$ za vsak $n\in \mathbb{N}$. Dokazujemo da ima  $\{a_n\}_{n\in \mathbb{N}}$ stekališče.
	\\
	$S = \{s; a_n < s$ kvečejmu za končno mnogo indeksov $n\}$.
	\\
	$S \ne \{ \emptyset\} \to $ če je $s<m , s\in S$.
	\\
	Naj obstaja $\alpha = \sup{S}$.
	Iz tega sledi, da je $S$ navzgor omejena: $s>M$. Iz tega pa sledi, da je $\alpha$ stekaliišče zaporedja.
\end{dokaz}

\begin{Definicija}
	Naj bo $\{a_n\}_{n\in \mathbb{N}}$ zaporedje realnih števil. Število $L\in \mathbb{R}$ je\textbf{ limita zaporedja} $\{a_n\}_{n\in \mathbb{N}}$, če zunaj vsake $\varepsilon$-okolice števila $L$ leži kvečejmu končno mnogo členov zaporedja.
	\\
	\\
	$L\in \mathbb{R}$ je limita zaporedja $\{a_n\}_{n\in \mathbb{N}}$, če za vsako $\varepsilon > 0$, obstaja $n_\varepsilon \in \mathbb{N}$, da za vsako  $n\in \mathbb{N}, n \ge n_\varepsilon$, velja $|a_n - L| < \varepsilon$.
\end{Definicija}

\begin{Definicija}
	Zaporedje $\{a_n\}_{n\in \mathbb{N}}$ je \textbf{konvergentno}, če ima limito.
\end{Definicija}

\begin{Opomba}
	Če je $L$ limita zaporedja, potem je $L$ tudi stekališče tega zaporedja. Obratno ni nujno res.
\end{Opomba}

\begin{dokaz}
	Denimo, da je $L$ limita zaporedja $\{a_n\}_{n\in \mathbb{N}}$. Dokazujemo, da je $L$ stekališče zaporedja $\{a_n\}_{n\in \mathbb{N}}$.
	\\
	Izberemo $\varepsilon > 0$. Ker je $L$ limita  $\{a_n\}_{n\in \mathbb{N}}$, leži zunaj intervala $(L -\varepsilon, L + \varepsilon)$, kvečejmo končno mnogo členov zaporedja. Vseh drugih, teh je neskončno mnogo ležijo znotraj intevala $(L -\varepsilon, L + \varepsilon)$. Iz tega sledi,da je $L$ stekališče zaporedja.
	
\end{dokaz}

\begin{Trditev}
	Če ima zaporedje več kot eno stekališče, potem ni konvergentno.
\end{Trditev}
\begin{dokaz}
	Naj bosta $s_1$ in $s_2$ različni stekališči zaporedja $\{a_n\}_{n\in \mathbb{N}}$.
	Dokazujemo, da $s_1$ ni limita zaporedja.
	\\
	Naj bo $d = \frac{1}{3}|s_1 - s_2|$.
	\\
	 Na intervalu $(s_2 - d, s_2 + d)$ leži neskončno mnogo členov zaporedja. Ker velja: $(s_1 -d, s_1+d) \cap (s_2 - d, s_2 + d) = \emptyset$, zato zunaj 	$(s_1 -d, s_1+d)$ leži končno mnogo členov zaporedja. Torej $s_1$ ni limita zaporedja. Simetrično tudi $s_2$ ni limita zaporedja. Nobeno število, ki ni stekališče, ni limita, torej zaporedje nima limite.
\end{dokaz}
\begin{Posledica}
	Konvergentno zaporedje ima natanko eno stekališče.
\end{Posledica}
\begin{dokaz}
	Limita zaporedja je tudi stekališče zaporedja in po trditvi je tudi edino stekališče.
\end{dokaz}
\begin{Izrek}
	Naraščajoče in navzgor omejeno zaporedje je konveregentno. Njegova limita je njegova natančna zgornja meja (supremum).
	\\
	Padajoče in navzdol omejeno zaporedje je konvergentno. Njegova limita je njegova natančna spodnja meja (infimum).
\end{Izrek}
\begin{dokaz}
	Naj bo $\{a_n\}_{n\in \mathbb{N}}$ naraščajoče in navzgor omejeno zaporedje.\\
	Naj bo $\alpha = \sup{\{a_n\}_{n\in \mathbb{N}}}$. Dokazujemo,da je $\alpha$ limita zaporedja.
	\\
	Ker je $\alpha$ natančna zgornja meja zaporedja, torej $\alpha - \varepsilon$ ni zgornja meja zaporedja.
	\\
	Torej obstaja: $a_n > \alpha - \varepsilon$
	\\
	$\alpha - \varepsilon < a_n \le a_{n+1} \le \alpha$
	\\
	$\alpha - \varepsilon < a_{n+k} < \alpha \quad \forall k \in \mathbb{N}$
	\\
	$a_{n+k} \in (\alpha - \varepsilon, \alpha)$
	\\
	Torej je $\alpha$ po definiciji, limita zaporedja.
\end{dokaz}
\begin{Opomba}
	Če je zaporedje konvergentno, potem je omejeno.
\end{Opomba}
\begin{dokaz}
	Denimo, da je $\{a_n\}_{n\in \mathbb{N}}$ konvergentno. Z $L$ označimo limito zaporedja.
	\\
	Zunaj intervala $(L - 1, L+ 1)$ leži kvečejmo končno mnogo členov zaporedja. Med njimi lahko izberemo najmanjše in največje število. To pomeni, da je zaporedje omejeno.
	\\
	Namesto $1$ bi lahko vzeli $\varepsilon > 0$.
\end{dokaz}
\begin{Definicija}
	Naj bosta $\{a_n\}_{n\in \mathbb{N}}$ in $\{b_n\}_{n\in \mathbb{N}}$ konvergentni zaporedji. Zaporedje $a_1 + b_1, a_2 + b_2,..., a_n + b_n, ... $ imenujemo vsota zaporedja $\{a_n\}_{n\in \mathbb{N}}$ in $\{b_n\}_{n\in \mathbb{N}}$ in označimo $\{a_n + b_n\}_{n\in \mathbb{N}}$.
	\\
	Podobno:
	\\
	razlika zaporedja  - $\{a_n - b_n\}_{n\in \mathbb{N}}$.
	\\
	produkt zapordja - $\{a_n \cdot b_n\}_{n\in \mathbb{N}}$.
	\\
	kvocientno zaporedje - $\{\frac{a_n}{b_n}\}_{n\in \mathbb{N}}$, če velja $b_n \ne 0, \forall n\in \mathbb{N}$.
\end{Definicija}
\begin{Izrek}
	Naj bosta $\{a_n\}_{n\in \mathbb{N}}$ in $\{b_n\}_{n\in \mathbb{N}}$ konvergentni zaporedji. Potem so zaporedja $\{a_n + b_n\}_{n\in \mathbb{N}}$,  $\{a_n - b_n\}_{n\in \mathbb{N}}$, $\{a_n \cdot b_n\}_{n\in \mathbb{N}}$ konvergentna zporedja in velja:
	$$\lim\limits_{n \to \infty}{(a_n + b_n)} = \lim\limits_{n \to \infty}{a_n} + \lim\limits_{n \to \infty}{b_n}$$
	$$\lim\limits_{n \to \infty}{(a_n - b_n)} = \lim\limits_{n \to \infty}{a_n} - \lim\limits_{n \to \infty}{b_n}$$
	$$\lim\limits_{n \to \infty}{(a_n \cdot b_n)} = \lim\limits_{n \to \infty}{a_n} \cdot \lim\limits_{n \to \infty}{b_n}$$
	Če velja $b_n \ne 0, \forall n\in \mathbb{N}$ in $\lim\limits_{n \to \infty}{b_n} \ne 0$, potem je zaporedje $\{\frac{a_n}{b_n}\}_{n\in \mathbb{N}}$ konvergentno in velja: $$\lim\limits_{n \to \infty}{\frac{a_n}{b_n}} = \frac{\lim\limits_{n \to \infty}{a_n}}{\lim\limits_{n \to \infty}{b_n}}$$
\end{Izrek}	
\begin{Posledica}
	Če je zaporedje  $\{a_n\}_{n\in \mathbb{N}}$ konvergentno in $c \in \mathbb{R}$, potem je \newline $\{c \cdot a_n\}_{n\in \mathbb{N}}$ konvergentno zaporedje in velja:
	$$\lim\limits_{n \to \infty}{(c \cdot a_n)} = c \cdot  \lim\limits_{n \to \infty}{a_n}$$
\end{Posledica}

\begin{Trditev}
	Naj bosta  $\{a_n\}_{n\in \mathbb{N}}$ in $\{b_n\}_{n\in \mathbb{N}}$ konvergentni zaporedji.
	\\
	\indent Če velja $a_n \le b_n$ za vse $n\in\mathbb{N}$, potem velja $\lim\limits_{n \to \infty}{a_n} \le \lim\limits_{n \to \infty}{b_n}$.
	\\
	\indent Če velja $a_n < b_n$ za vse $n\in\mathbb{N}$, potem velja $\lim\limits_{n \to \infty}{a_n} \le \lim\limits_{n \to \infty}{b_n}$.
	\\
	\indent Opomba: V limiti se lahko pojavi enačaj!
\end{Trditev}
\begin{Definicija}
	Naj bo $\{a_n\}_{n\in \mathbb{N}}$ zaporedje in naj bo $\{p\}_{n\in \mathbb{N}}$ strogo naraščajoče zaporedje naravnih števil. Potem imenujemo zaporedje $b_n = a_{p_n}$ \textbf{podzaporedje} zaporedja $\{a_n\}_{n\in \mathbb{N}}$.
\end{Definicija}
\begin{Izrek}
	Naj bo $\{a_n\}_{n\in \mathbb{N}}$ zaporedje in $s\in \mathbb{R}$ stekališče zaporedja $\{a_n\}_{n\in \mathbb{N}}$. Potem obstaja konvergentno pozaporedje zaporedja $\{a_n\}_{n\in \mathbb{N}}$, ki konvergira proti $s$.
\end{Izrek}
\begin{dokaz}
	Naj bo $s$ stekališče $\{a_n\}_{n\in \mathbb{N}}$.
	Obstaja : $a_{n_1} \in (s -1, s + 1)$
	\\
	$a_{n_2}\in (s - \frac{1}{2}, s +  \frac{1}{2})$ in $n_2 > n_1$, zato ker je na intervalu $(s - \frac{1}{2}, s +  \frac{1}{2})$ neskončno mnogo členov zaporedja, samo končno mnogo jih ima premajhen indeks.
	\\
	Nadaljujemo na enak način in dobimo podzaporedje:
	$a_{n_1}, a_{n_2}, a_{n_3},\dots$, ker so \newline $n_1 < n_2 < n_3 < \dots$ in $ a_{n_k} \in (s - \frac{1}{k}, s +  \frac{1}{k})$, velja:
	$$\lim\limits_{n \to \infty}{a_{n_k}} = s.$$
\end{dokaz}
\begin{Izrek}
	Naj bo  $\{a_n\}_{n\in \mathbb{N}}$ zaporedje. Če je $L$ limita konvergentnega podzaporedja zaporedja  $\{a_n\}_{n\in \mathbb{N}}$, potem je $L$ stekališče zaporedja  $\{a_n\}_{n\in \mathbb{N}}$.
\end{Izrek}
\begin{dokaz}
	Denimo, da je $L$ limita podzaporedja. Potem $\forall \varepsilon > 0$ na $(L - \varepsilon, L + \varepsilon)$ ležijo vsi, razen kvečejmu končno mnogo členov podzaporedja. Torej na intervalu $(L - \varepsilon, L + \varepsilon)$ leži neskončno členov zaporedja. Zato je $L$ stekališče.
\end{dokaz}
\begin{Posledica}
	Vsako omejeno zaporedje ima konvergentno podzaporedje.
\end{Posledica}
\begin{dokaz}
	Vsako omejeno zaporedje ima stekališče $s$. Po prejšnem izreku obstaja podzaporedje, ki konvergira k $s$.
\end{dokaz}
\begin{Trditev}
	Naj bo $L$ limita zaporedja  $\{a_n\}_{n\in \mathbb{N}}$.
	$$\lim\limits_{n \to \infty}{a_n} = L$$
	Tedaj vsako podzaporedje  $\{a_{n_j}\}_{j\in \mathbb{N}}$ konvergira k $L$.
\end{Trditev}
\begin{Trditev}
	Naj bo  $\{a_n\}_{n\in \mathbb{N}}$ zaporedje. Tedaj velja:
	$$\lim\limits_{n \to \infty}{a_n} = 0 \iff \lim\limits_{n \to \infty}{|a_n|} = 0$$
\end{Trditev}
\begin{Trditev}
	Naj za zaporedja  $\{a_n\}_{n\in \mathbb{N}}$,  $\{b_n\}_{n\in \mathbb{N}}$,  $\{c_n\}_{n\in \mathbb{N}}$, velja $ b_n <a_n < c_n$.
	Iz tega sledi, da če je:
	$$ \lim\limits_{n \to \infty}{b_n} = 0 \quad   in  \quad \lim\limits_{n \to \infty}{c_n} = 0 \ \Rightarrow \lim\limits_{n \to \infty}{a_n} = 0$$
\end{Trditev}

\begin{Definicija}
	Zaporedje  $\{a_n\}_{n\in \mathbb{N}}$ je \textbf{Cauchijevo}, če za vsak $\varepsilon > 0, \exists n_0$,\newline$ n,m >n_0$, velja $|a_n - a_m| < \varepsilon$.
\end{Definicija}
\begin{Izrek}
	Zaporedje $\{a_n\}_{n\in \mathbb{N}}$ je Cauchijevo $\iff$ ko je konvergentno.
\end{Izrek}

\begin{Definicija}
	Zaporedje $\{a_n\}_{n\in \mathbb{N}}$ konvergira k  $ + \infty$, če za vsak $M, \exists n_0, \forall n \ge n_0$ velja $M < a_n$. (Tako zaporedje NE konvergira).
	\\
	\indent Zaporedje $\{a_n\}_{n\in \mathbb{N}}$ konvergira k  $ - \infty$, če za vsak $M, \exists n_0, \forall n \ge n_0$ velja $M > a_n$.
\end{Definicija}

\pagebreak
\section{Številske vrste}

\begin{Definicija}
	Pravimo,da je številska vrsta $\sum_{k = 1}^{\infty}{a_k}$ \textbf{konvergentna}, če je konvergentno pripadajoče zaporedje delnih vsot  $\{s_n\}_{n\in \mathbb{N}}$.
	Vsota vrste je limita zaporedja delnih vsot, če obstaja. Če zaporedje  $\{s_n\}_{n\in \mathbb{N}}$ divergira, pravimo, da vrsta divergira.
\end{Definicija}
\begin{Trditev}
	Če vrsta $\sum_{n = 1}^{\infty}{a_n}$ konvergira, potem zaporedje  $\{a_n\}_{n\in \mathbb{N}}$ konvergentno in ima $\lim\limits_{n \to \infty}{a_n} = 0$.
	\\
	Obratno ne velja!
\end{Trditev}
\begin{dokaz}
	Denimo, da vrsta $\sum_{n = 1}^{\infty}{a_n}$ konvergira. Torej zaporedje delnih vsot  $\{s_n\}_{n\in \mathbb{N}}$ konvergira.
\end{dokaz}
\begin{Trditev}
	Denimo, da sta vrsti $\sum_{n = 1}^{\infty}{a_n}$ in $\sum_{n = 1}^{\infty}{b_n}$ konvergentni in $c \in \mathbb{R}$. Potem so tudi vrste $\sum_{n = 1}^{\infty}{(a_n+ b_n)}$, $\sum_{n = 1}^{\infty}{(a_n - b_n)}$  in $\sum_{n = 1}^{\infty}{c \cdot a_n}$ konvergentne. 
\end{Trditev}

\begin{dokaz}
	Naj bosta $s_n := a_1 + a_2 + \dots + a_n$ in $t_n := b_1 + b_2 + \dots + b_n$ zaporedje delnih vsot. Ker sta vrsti konvergentni, sta $\{s_n\}$ in $\{t_n\}$ konvergentni. Dokazujemo, da je vrsta $\sum_{n = 1}^{\infty}{(a_n+ b_n)}$ konvergentna.
	\\
	Njeno zaporedje delnih vsot $u_n = (a_1 + b_1) + (a_2 + b_2) + \dots + (a_n + b_n) = s_n + t_n$ Ker sta $s_n$ in $t_n$ konvergentni zaporedji je tudi $u_n$ konvergentno zaporedje. Iz tega lahko sklepamo, da je tudi vsota $\sum_{n = 1}^{\infty}{(a_n+ b_n)}$ konvergentna.
\end{dokaz}
\begin{zgled}
	Vrsta $ 1 + \frac{1}{2} + \frac{1}{3} + \dots + \frac{1}{n} + \dots = \sum_{n = 1}^{\infty}$ se imenuje \textbf{harmonična vrsta}. Harmonična vrsta divergira.
	\\
	\\
	Vrsta $a + aq + aq^2 +aq^3 + \dots + aq^n + \dots = \sum_{n = 1}^{\infty} \quad a\in \mathbb{R}\backslash \{0\}, \quad q\in \mathbb{R} \backslash \{1\}$ imenujemo \textbf{geometrijska vrsta}. Geometrijska vrsta konvergira natanko tedaj ko je $|q| < 1$, potem je njena vsota $ \frac{a}{1 - q}$.
\end{zgled}
\subsection{Kriteriji za konvergenco vrst}

\begin{Izrek}
 	(Cauchyjev kriterij) Naj bo $\sum_{n = 1}^{\infty}{a_n}$ dana številska vrsta. $\sum_{n = 1}^{\infty}{a_n}$ konvergira natanko tedaj kadar za vsak $\varepsilon > 0$, obstaja $n_\varepsilon \in \mathbb{N}$, da za vsak $n,m > n_\varepsilon, n,m\in \mathbb{N}$ velja:
 	$$ |a_m + a_{m+1} + \dots + a_n| < \varepsilon$$
\end{Izrek}
\begin{Opomba}
	Pogoj na desni se imenuje Cauchijev pogoj. Spomni se na Cauchyjevo zaporedje.
\end{Opomba}
\begin{Izrek}
	(Primerjalni kriterij 1) Naj bo $\sum_{n = 1}^{\infty}{c_n}$ konvergentna vrsta z nenegativnimi členi. Če velja $$0 \le a_n \le c_n$$ za vsak $n\in \mathbb{N}$, potem $a_n$ konvergira.
\end{Izrek}
\begin{dokaz}
	Naj bo $t_n = c_1 + c_2 + \dots + c_n$ in naj bo $s_n = a_1 + a_2 + \dots + a_n$.
	\\
	Ker imamo vrste z nenegativnimi členi, sta $\{t_n\}_{n\in \mathbb{N}}$ in $\{s_n\}_{n\in \mathbb{N}}$ naraščajoči zaporedji. Ker je $\sum_{n =1}^{\infty}{c_n}$ konvergentna, je tudi $\{t_n\}_{n\in \mathbb{N}}$ konvergentno zaporedje.
	\\
	Ker je $ \le a_n \le c_n \quad   \forall n$,  velja: $0 \le s_n \le t_n \quad \forall n$.
	\\ 
	Ker je $\{t_n\}$ konvergentno je tudi omejeno : $t_n \le M \quad \forall n$. Iz zgornje neenakosti sledi: $ s_n \le M \quad \forall n$. Kar pa pomeni, da je $\{s_n\}$ konvergentno zaporedje, saj je naraščajoče in omejeno.
\end{dokaz}
\begin{Izrek}
	(Primerjalni kriterij 2) Naj bo $\sum_{n = 1}^{\infty}{d_n}$ divergentna vrsta z nenegativnimi členi in če velja $$ 0 \le d_n \le b_n$$, potem je tudi $\sum_{n = 1}^{\infty}{b_n}$ divergentno.
\end{Izrek}
\begin{dokaz}
	$\forall n \quad  0 <d_1 + d_2 + \dots + d_n < b_1 + b_2 + \dots + b_n$.
	\\
	Ker $\sum_{n = 1}^{\infty}{d_n}$ divergira in ima pozitivne člene, je zaporedje njenih delnih vsot neomejeno. Zato je tudi zaporedje $\{b_1, b_2, \dots , b_n\}$ neomejeno.
\end{dokaz}
\begin{Izrek}
	(Kvocientni kriterij) Naj bo $\sum_{n = 1}^{\infty}{a_n}$ vrsta s pozitivnimi členi. Denimo da je zaporedje $d_n = \frac{a_{n+1}}{a_n}$ konvergira in ima limito $d$.
	\newline
	Če je  $d < 1$, potem $\sum_{n = 1}^{\infty}{a_n}$ konvergira
	\newline
	Če je $d > 1$, potem $\sum_{n = 1}^{\infty}{a_n}$ divergira.
	\newline
	Če je $d = 1$, potem ne vemo nič.
\end{Izrek}
\begin{Izrek}
	(Korenski kriterij) Naj bo $\sum_{n = 1}^{\infty}{a_n}$ vrsta s pozitivnimi členi. Denimo, da je zaporedje $c_n =\sqrt[n]{a_n}$ konvergentno z limito $c$.
	\newline
	Če je $ c > 1$, vrsta $\sum_{n = 1}^{\infty}{a_n}$ divergira.
	\newline
	Če je $ c < 1$, vrsta $\sum_{n = 1}^{\infty}{a_n}$ konvergira.
	\newline
	Če je $c = 1$, ne vemo nič.
\end{Izrek}
\begin{Definicija}
	Pravimo, da je vrsta $\sum_{n = 1}^{\infty}{a_n}$ \textbf{alternirajoča}, če je $a_n \cdot a_{n+1} \le 0$ za vsak $n\in \mathbb{N}$.
\end{Definicija}
\begin{Opomba}
	Če je $\{b_n\}_{n\in \mathbb{N}}$ zaporedje pozitivnih števil, potem je vrsta alternirajoča:
	$$\sum_{n = 1}^{\infty}{(-1)^n b_n}$$
\end{Opomba}
\begin{Izrek}
	(Leibnizev kriterij) Naj bo $\{a_n\}_{n\in \mathbb{N}}$, padajoče zaporedje, nenegativnih števil z limito $0$. Potem $\sum_{n = 1}^{\infty}{(-1)^n a_n}$ konvergira.
\end{Izrek}
\begin{Definicija}
	Naj bo $\sum_{n = 1}^{\infty}{a_n}$ številska vrsta. Pravimo, da $\sum_{n = 1}^{\infty}{a_n}$ \textbf{absolutno konvergira}, če konvergira $\sum_{n = 1}^{\infty}{|a_n|}$. Če vrsta konvergira, in divergira absolutno, potem pravimo, da \textbf{konvergira pogojno}.
\end{Definicija}
\begin{Trditev}
	Naj bo $\sum_{n = 1}^{\infty}{a_n}$ številska vrsta. Če je vrsta absolutno konvergentna, potem je konvergentna.
\end{Trditev}

%%%%%%%%%%%%%%%%%%%%%%%%%%%%%%%%%%%%%%%FUNKCIJE%%%%%%%%%%%%%%%%%%%%%%%%%%%%%%%%%%%%%%%%%%%%%%%%%%%%%
\section{Funkcije}
\begin{Definicija}
	Funkcija ali preslikava $f: A \to B$ je predpis, ki vsakemu elementu $a \in A$ priredi natanko en element $f(a) \in B$. Množica $A$ imenujemo \textbf{definicijsko območje} funkcije f ali \textbf{domena}, množico $B$ pa imenujemo \textbf{zaloga vrednosti} ali \textbf{kodomena}.
\begin{Opomba}
	Zaloga vrednosti funkcije $f$ je množica vseh elementov iz $B$, ki so slike kakšnega elementa iz množice $A$:
	$$ Z_f = \{f(a); a\in A \} \subset B$$
\end{Opomba}
\end{Definicija}
\begin{Definicija}
	Naj bo $f: D \to \mathbb{R}$ funkcija.\textbf{ Graf funkcije f}, je množica urejenih parov:
	$$\Gamma_f = \{(x,f(x), x\in D)\}$$
\end{Definicija}
\begin{Definicija}
	Naj bo funkcija $f: D\to \mathbb{R}$ definirana na intervalu D. Pravimo, da je f \textbf{naraščajoča} na D, če velja: $ \forall x_1, x_2 \in D$ in $x_1 \le x_2$, potem \newline $f(x_1) \le f(x_2)$.
	\newline
	Funkcija je \textbf{strogo naraščajoča}, če za vsak $x_1, x_2 \in D$ velja $x_1 < x_2$, potem $f(x_1) < f(x_2)$.
\end{Definicija}
\begin{Definicija}
	Naj bo funkcija $f: D\to \mathbb{R}$ definirana na intervalu D. Pravimo, da je f \textbf{padajoča} na D, če velja: za vsak $x_1, x_2 \in D$ in $x_1 \le x_2$, potem \newline $f(x_1) \ge f(x_2)$.
	\newline
	Funkcija je \textbf{strogo padajoča}, če za vsak $x_1, x_2 \in D$ velja $x_1 < x_2$, potem $f(x_1) > f(x_2)$.
\end{Definicija}
\begin{Opomba}
	Funkcija je \textbf{monotona} na D, če je bodisi naraščajoča bodisi padajoča.
\end{Opomba}
\begin{Definicija}
	Naj  bo D interval in $f: D \to \mathbb{R}$ funkcija. Zgornja meja funkcije f na D je $M\in \mathbb{R}$, z lastnostjo, da je $f(x) \le M$ za vsak $x \in D$.
	\textbf{Funkcija je navzgor omejena} na D, kadar ima $f$ na D kakšno zgornjo mejo.
	Če je f navzgor omejena na D, imenujemo število $\sup\{f(x),x\in D\}$ natančna zgornja meja funkcije f.
	\newline
	Podobno definiramo spodnjo mejo $f$ na $D$, navzdol omejeno in natančno spodnjo mejo.
\end{Definicija}
\begin{Definicija}
	Naj bo $f: D\to \mathbb{R}$ definirana na simetričnem intervalu D.
	\newline
	Funkcija f je  \textbf{soda}, če velja $f(x) = f(-x)$ za vsak $x \in D$.
	\newline
	Funkcija f je \textbf{liha}, če velja $f(x) = -f(x)$ za vsak $x \in D$.
\end{Definicija}
\begin{Opomba}
	Graf sode funkcije je simteričen glede na ordinatno os.
	\newline
	Graf lihe funkcije je simetričen glede na izhodišče koordinatnega sisitema.
\end{Opomba}
\begin{Definicija}
	Naj bo $f: A \to B$ preslikava. Preslikava $f$ je \textbf{injektivna}, če poljubna različna elementa množice $A$ preslika v različne elementa množice $B$.
	$$f(x) = f(y) \Rightarrow x = y$$
	\\
	Preslikava $f$ je \textbf{surjektivna}, če velja, da je zaloga vrednosti enaka celotni množici $B$. 
	$$Z_f = B$$	
	\\
	Preslikava je \textbf{bijektivna}, natanko tedaj kadar je injektivna in surjektivna.
\end{Definicija}
\begin{Definicija}
	Funkcijo $f: A \times A \to A$, imenujemo tudi dvočlena operacija na množici $A$.
\end{Definicija}
\begin{Definicija}
		Naj bosta $f: A \to B$ in $g: B\to C$ preslikavi. Preslikava $g \circ f: A \to C$, ki je definirana s predpisom
		$$ a \mapsto g(f(a))$$
		imenujemo \textbf{kompozitum preslikav} $f$ in $g$.
\end{Definicija}
\begin{Definicija}
	Naj bo $f: A \to B$ preslikava. Če je $f$ bijektivna preslikava, potem $f^{-1}: B\to A$  \textbf{inverzna preslikava}, ki jo definiramo:
	$$f^{-1}(b) = a \iff f(a) = b$$
\end{Definicija}
\begin{Posledica}
		$$a\in A :  f^{-1}(f(a)) = f^{-1}(b) = a \Rightarrow f^{-1} \circ f \equiv id_A$$
		$$b\in B :  f(f^{-1}(b)) = f(a) = b  \Rightarrow f \circ f^{-1} \equiv id_B$$
\end{Posledica}
\begin{Definicija}
	Naj bosta $A$ in $B$ množici. Pravimo, da imata $A$ in $B$ \textbf{isto moč}, če obstaja bijektivna preslikava $f: A \to B$. ($A$ in $B$ sta ekvivalentni).
\end{Definicija}
\begin{Opomba}
	Če je $f:A \to B$ bijektivna je tudi $f^{-1}: B\to A$ bijektivna.
\end{Opomba}
\begin{Opomba}
	Končni množici imata isto moč natanko tedaj, kadar imata isto število elementov.
\end{Opomba}
\begin{Definicija}
	Množica je \textbf{števno neskončna}, če imata $A$ in $\mathbb{N}$ isto moč. Neskončna množica, ki ni števno neskončna, je neštevna.
\end{Definicija}
\begin{Opomba}
	Denimo, da je $A$ števno neskončna. \newline Potem obstaja bijekcija  $A \to \mathbb{N}$.
	\\
	$n \in \mathbb{N}: \quad  n\mapsto  f(n) = a_n$
	\\
	$A = \{a_1, a_2, a_3, \dots, a_n\}$
\end{Opomba}
\begin{zgled}
	$ S := $ množica sodih naravnih števil.
	\\
	S je števno neskončna, saj obstaja bijektivna preslikava $f: \mathbb{N} \to S$, ki je definirana s predpisom: $n \mapsto 2n$.
\end{zgled}
%%%%%%%%%%%%%%%%%%%%LIMITE%%%%%%%%%%%%%%%%%%%%%%%%%%%%%%%%%%%%%%%%%%%%%%%%%%%%%%%%%%%%
\subsection{Limite in zveznost}
\begin{Definicija}
	Pravimo, da je $L\in\mathbb{R}$ \textbf{limita funkcije} $f$ v točki $a$, če za vsako konvergentno zaporedje $\{a_n\}_{n\in \mathbb{N}}$ z limito $a_n$ velja, da je $\{f(a_n)\}_{n\in \mathbb{N}}$ konvergentno z limito $L$.
\end{Definicija}
\begin{Trditev}
	Naj bosta g in f definirani v okolici točke a, razen morda v a. Če obstaja $ \lim\limits_{x\to a}{f(x)}$ in $ \lim\limits_{x\to a}{g(x)}$, potem velja:
	\begin{enumerate}
		\item
		$\lim\limits_{x \to a}{ \lambda f(x)} = \lambda  \lim\limits_{x\to a}{f(x)}$
		\item
		$\lim\limits_{x \to a}{(f(x) \pm g(x))} =  \lim\limits_{x\to a}{f(x)} \pm \lim\limits_{x\to a}{g(x)}$
		\item
		$ \lim\limits_{x\to a}{(f(x)\cdot g(x) )} =  \lim\limits_{x\to a}{f(x)} \cdot  \lim\limits_{x\to a}{g(x)}$
		\item
		$\ \lim\limits_{x\to a}{\frac{f(x)}{g(x)}} = \frac{ \lim\limits_{x\to a}{f(x)}}{ \lim\limits_{x\to a}{g(x)}}, \quad  \lim\limits_{x\to a}{g(x)} \ne 0\quad  in \quad  g(x) \ne 0$
		\item
		$f(x) \le g(x) \iff  \lim\limits_{x\to a}{f(x)} \le  \lim\limits_{x\to a}{g(x)}$
	\end{enumerate}
\end{Trditev}
\begin{Trditev}
	Naj bo f definirana v okolici točke a razen morda v a. Potem velja:
	$$L =  \lim\limits_{x\to a}{f(x)} \iff \forall \varepsilon > 0, \quad \exists \delta > 0,$$ da velja $ 0 < |x - a| < \delta$, potem velja: $|f(x) - L | < \varepsilon$.
\end{Trditev}
\begin{Definicija}
	Naj bo funkcija f definirana na intervalu $(a, a + \varepsilon)$ za nek $\varepsilon > 0$. Število $L\in \mathbb{R}$, je \textbf{desna limita} funkcije f v točki a, če za vsak $\varepsilon > 0$, obstaja $\delta >0$, da za $x \in (a, a+\delta)$, potem velja: $|f(x) - L| < \varepsilon$.
	$$L = \lim\limits_{x \searrow a}{f(x)}$$
\end{Definicija}
\begin{Definicija}
	Naj bo funkcija f definirana na intervalu $(a-\varepsilon, a)$ za nek $\varepsilon > 0$. Število $L\in\mathbb{R}$, je \textbf{leva limita} funkcije f v točki a, če za vsak $\varepsilon > 0,$ obstaja $ \delta > 0$, da za $x\in(a- \delta, a)$, velja $|f(x) - L | < \varepsilon$.
	$$L = \lim\limits_{x \nearrow a}{f(x)}$$
\end{Definicija}
\begin{Definicija}
	Naj bo $(a, \infty)$ interval in $f: (a, \infty) \to \mathbb{R}$ funkcija. Pravimo, da je $L\in \mathbb{R}$ \textbf{limita funckcije f v plus neskončnost}, če za vsak $\varepsilon > 0$, obstaja $M\in \mathbb{R}$, da za vsak $ x > M (M > a)$ velja $f(x) \in (L - \varepsilon, L + \varepsilon)$
	$$ L = \lim\limits_{x\to \infty}{f(x)}$$
\end{Definicija}
\begin{Definicija}
	Naj bo $(-\infty, b)$ interval in $f: (-\infty, b) \to \mathbb{R}$ funkcija. Pravimo, da je $L\in \mathbb{R}$ \textbf{limita funkcije v minus neskončnost}, če za vsak $\varepsilon > 0$, obstaja $m \in \mathbb{R}$, da za vsak $x < m (m < b)$ velja, $f(x) \in (L -\varepsilon, L +\varepsilon )$.
	$$ L = \lim\limits_{x \to -\infty}{f(x)}$$
\end{Definicija}
\begin{Definicija}
	Naj bo funkcija definirana v okolici točke $a$, razen morda v $a$. Pravimo, da ima funkcija v točki $a$ limito neskončno, če za vsako še tako veliko število $M\in \mathbb{R}$, obstaja $\delta > 0$, da velja:
	\\
	če je $x \in (a - \delta, a+ \delta)\backslash \{a\}$, potem $f(x) > M$.
\end{Definicija}
%%%%%%%%%%%%%%ZVZENOST%%%%%%%%%%%%%%%%%%%%%%%%%%%%%%%%%%%%
\begin{Definicija}
	Naj bo funkcija definirana v okolici točke $a$. Pravimo, da je $f$ \textbf{zvezna v točki a}, če za vsak $\varepsilon > 0$, obstaja $\delta > 0$, da za x: $ |x - a| < \delta$, velja: $|f(x) - f(a) | < \varepsilon$.
\end{Definicija}
\begin{Trditev}
	Naj bo funkcija $f$ definirana v okolici točke a. Potem je funkcija f zvezna v točki natanko tedaj, kadar obstaja $\lim\limits_{x \to a}{f(x)}$  in je enaka $f(a)$.
\end{Trditev}
\begin{dokaz}
	$(\Rightarrow)$ če je $f$ zvezna v točki $a$, potem lahko za $L$ vzamemo $f(a)$.
	\\
	$(\Leftarrow)$ če limita obstaja in je enaka $f(a)$, potem je $f$ zvezna v $a$.
\end{dokaz}
\begin{Definicija}
	Naj bo funkcija $f$ definirana na $(a,b)$. Potem je $f$ zvezna $(a,b)$, če je zvezna v vsaki točki $x_0 \in (a,b)$.
	\\
	Pravimo tudi, da je $f$ \textbf{zvezna funkcija}, če je zvezna na svojem definicijskem območju.
\end{Definicija}
\begin{Trditev}
	Naj bosta $f$ in $g$ zvezni v točki a. Potem so tudi funkcije $f+g$, $f -g$ in $f\cdot g$ zvezne v točki $a$.
	Če je $g(a) \ne 0$, potem je tudi $\frac{f}{g}$ zvezna v točki $a$.
\end{Trditev}
\begin{Trditev}
	Naj bo funkcija f definirana v okolici točke $y_0$ in zvezna v točki $y_0$. Naj bo funkcija $y = y(x)$ definirana v okolici $x_0$ in zvezna v točki $x_0, \quad y(x) = y_0$. Potem je funkcija $f\circ g$ definirana v okolici točke $x_0$ in zvezna v točki $x_0$.
	$$\lim\limits_{x \to x_0}{f(y(x))} = f(\lim\limits_{x \to x_0}{y(x)})$$
\end{Trditev}
\begin{Posledica}
	Naj bo $\{x_n\}_{n\in \mathbb{N}}$ konvergentno zaporedje z limito $L$ in naj bo funkcija $f$ definirana v okolici točke $L$ in zvezna v $L$.
	$$\lim\limits_{n\to \infty}{f(x_n)} = f (\lim\limits_{n\to \infty }{x_n}) = f(L)$$
\end{Posledica}
\begin{Izrek}
	Vse elementarne funkcije so zvezne.
\end{Izrek}
\begin{Izrek}
	(Bisekcija) Naj bo funkcija f definirana in zvezna na $[a,b]$. Če je $f(a) \cdot f(b) < 0$, potem ima f na $[a,b]$ ničlo. To pomeni, da obstaja $c\in (a,b)$, za katero velja $f(c) = 0$.
\end{Izrek}
\begin{Definicija}
	Funkcija $f: D_f \to \mathbb{R}$ je \textbf{omejena}, kadar je $Z_f$ omejena. V tem primeru obstajata $m,M \in \mathbb{R}$, da velja $ m\le f(x)\le M$, za vsak $x\in D_f$.
\end{Definicija}
\begin{Izrek}
	Naj bo f zvezna funkcija na $[a,b]$. Potem je funkcija f omejena.
\end{Izrek}
\begin{Izrek}
	Naj bo $f$ zvezna funkcija na $[a,b]$. Potem funkcija $f$ doseže minimum ali maksimum, to pomeni, da obstaja $x_m, x_M \in [a,b]$, za katerega velja:
	$$f(x_m) = \min\{f(x); x\in [a,b]\}$$
	$$f(x_M) = \max\{f(x); x\in [a,b]\}$$
\end{Izrek}
\begin{Izrek}
	Naj bo $f$ zvezna funkcija na $[a,b]$. Potem $f$ doseže vse vrednosti med $\min$ in $\max$.
\end{Izrek}
%%%%%%%%%%%%%%%%%%%%%%%%%%%%ODVOD%%%%%%%%%%%%%%%%%%%5
\section{Odvod}
\begin{Definicija}
	Naj bo funkcija $f$ definirana v okolici točke a. Če obstaja
	$$ \lim\limits_{h \to 0}{\frac{f(a+h) - f(a)}{h}}$$ potem jo imenujemo \textbf{odvod funkcjije f} v točki a in označimo $f'(a)$.
\end{Definicija}
\begin{Definicija}
	Naj bo f definirana na $[a, a+\varepsilon]$ za $\varepsilon > 0$. \textbf{Desni odvod} funkcije f v točki $a$ je
	$$ \lim\limits_{h \searrow 0}{\frac{f(a+h) - f(a)}{h}}$$
\end{Definicija}
\begin{Definicija}
	Naj bo f definirana na $[a, a+\varepsilon]$ za $\varepsilon > 0$. \textbf{Levi odvod} funkcije f v točki $a$ je
	$$ \lim\limits_{h \nearrow 0}{\frac{f(a+h) - f(a)}{h}}$$
\end{Definicija}
\begin{Trditev}
	Naj bo funkcija f definirana v okolici točke $a$. Potem je f odvedljiva v točki a natanko tedaj, kadar obstajata levi in desni odvod in sta enaka.
\end{Trditev}
\begin{Izrek}
	Naj bo funkcija f definirana v okolici točke a. Če je f odvedljiva v točki a, potem je zvezna v točki a.
\end{Izrek}
\begin{Opomba}
	Obratno ne velja. Primer: $f(x) = |x|$ v $f(0) = 0$
\end{Opomba}
\begin{Definicija}
	Naj bo funkcija f definirana na $(a,b)$. Pravimo, da je f odvedljiva na $(a,b)$, če je f odvedljiva v vsaki točki na intervalu $(a,b)$. Če je f odvedljiva na $(a,b)$, potem je $f'$ funkcija, ki je definirana na $(a,b)$. Če je $f'$ zvezna na $(a,b)$, rečemo, da je $f$ \textbf{zvezno odvedljiva} na $(a,b)$.
\end{Definicija}

\begin{Definicija}
	Naj bo funkcija f odvedljiva na $(a,b)$. Njen odvod je funkcija, ki je definirana na $(a,b)$. Če je ta nova funkcija $(f')$ odvedljiva na $(a,b)$, jo lahko še enkrat odvajamo in dobimo \textbf{drugi odvod funkcije} $f$, kar označimo s $f''$.
\end{Definicija}
\begin{Definicija}
	Funkcija f, ki je definirana v okolici točke $a$, je \textbf{diferenciabilna} v točki a, če obstaja taka linearna funkcija ${L} : h \to {L}(h)$ za katero velja:
	$$ \triangle f = f(a+h) - f(a) = L(h) + o_a(h)$$
	kjer, velja da je $\lim\limits_{h\to 0}{\frac{o_a(h)}{h}} = 0$.
\end{Definicija}
\begin{Opomba}
	Vsaka linearna funkcija $L: \mathbb{R} \to \mathbb{R}$ je oblike $ h\mapsto kh$, za nek $k\in \mathbb{R}$. Izplejali smo, da če je f odvedljiva v točki $a$, potem je ff diferenciabilna v točki a in $L(h) = f'(a) \cdot h$ in $o_a(h) = o(h)$.
\end{Opomba}
\begin{Definicija}
	Funkcija L iz prejšne definicije imenujemo \textbf{diferencial funkcije f} v točki a. Če je f odvedljiva v točki a, potem velja:
	$$ L(h) = f'(a) \cdot h$$
	\\
	Oznaka : $df_a(h) = f'(a) \cdot h $
\end{Definicija}
\begin{Izrek}
	Naj bo funkcija f definirana v okolici  točke a. Potem je f odvedljiva v točki a natanko tedaj, kadar je f diferenciabilna v točki a.
	V tem primeru velja: $ d f_a(h) = f'(a) \cdot h$
\end{Izrek}
\begin{Trditev}
	\begin{itemize}
		\item
			Odvod konstante je $0$.
		\item
			Odvod indetitete je $1$.
			\\
			$f(x) = x \quad \forall x\in \mathbb{R}$
			\\
			$x\in \mathbb{R}: \quad f'(x) = \lim\limits_{h \to 0}{\frac{f(h+x) - f(x)}{h}} = \lim\limits_{h\to 0}{\frac{h + x -x}{h}} = \lim\limits{h \to 0}{1} = 1$
		\item
			Odvod vsote funkcije je vsota odvod. Isto za razliko.
			$$(f+g)' = f' + g'$$
			\\
			\\
			$(f+g)'(a) = \lim\limits_{h \to 0}{\frac{f+g(h+a) - f+g(a)}{h}} = \lim\limits_{h \to 0}{\frac{f(h+a)+g(h+a) - f(a) + g(a)}{h}}  = \lim\limits_{h\to 0}{\frac{f(a+h)- f(a)}{h}} + \lim\limits_{h \to 0}{\frac{g(h+a) - g(a)}{h}} = f' + g'$
		\item
			Odvod produkta funkcij:
			$$(f\cdot g)' = f' \cdot g + f\cdot g' $$
		\item
			Odvod kvocienta odvedljivih funkcij:
			$$( \frac{f}{g})' = \frac{f' \cdot g - g' \cdot f}{g^2}$$
		\item
			Odvod sestavljene funkcije:
			$$(f\circ g)' = (f' \circ g)\cdot g'$$
		\item
			Odvod inverzne funkcije:
			$$(f^{-1})' = \frac{1}{f' \circ f^{-1}}$$
	\end{itemize}
\end{Trditev}
\begin{Definicija}
	Naj bo funkcija f definirana v okolici točke $a$. Če velja $f(a) \ge f(x) \quad \forall x$ v neki okolici točke a, potem ima f v točki a \textbf{lokalni maksimum}.
	\\
	Če velja $f(a) \le f(x) \quad \forall x$ v neki okolici točke a, potem ima f v točki a \textbf{lokalni minimum}.
\end{Definicija}
\begin{Izrek}
	Naj bo funkcija f definirana v okolici točke a in naj bo f odvedljiva v a. Če ima f v točki a lokalni ekstrem potem je $f'(a) = 0$.
\end{Izrek}
\begin{Izrek}
	(\textbf{Rollov}) Naj bo f zvezna funkcija na $[a,b]$ in odvedljiva na $(a,b)$. Če je $f(a) = f(b)$, potem obstaja $c\in (a,b)$, da velja $f'(c) = 0$.
\end{Izrek}
\begin{dokaz}
	Ker je f zvezna na zaprtem intervalu $[a,b]$ doseže globalni maksimum in globalni minimum. 
	\newline
	Prva možnost : m = M $\to$ f je konstanta $f'(x) = 0$ OK. (m in M sta globalna ekstrema)
	\newline
	Druga možnost: $ m < M$:
		Ker velja, $f(a) = f(b)$, leži vsaj eno od števil $x_m$ in $x_M$ v notranjosti intervala $[a,b]$. Recimo, da je to $x_M$. Potem je $x_M$ lokalni minimum od f. Po izreku velja: $f'(x_M) = 0$.
		(ni čist kul(glej recimo.))
\end{dokaz}
\begin{Izrek}
	(\textbf{Lagrangeev izrek}) Naj bo f zvezna funkcija na $[a,b]$ in odvedljiva na $(a,b)$. Potem obstaja $c\in (a,b)$, za katerega velja:
	$$f'(c) = \frac{f(b)- f(a)}{b -a}$$
\end{Izrek}

\begin{dokaz}
	Definirajmo funkcijo $g: [a,b]\to \mathbb{R}$ s predpisom:\newline $g(x) =f(x) - \frac{f(b) -f(a)}{b - a}(x-a)$. $g$ je zvezna na $[a,b]$ in odvedljiva na $(a,b)$.
	\newline
	$g(a) = f(a)$
	\newline
	$g(b) =f(a) - (f(b) - f(a)) = f(a)$
	\newline
	Po Rollovem izreku velja, da obstaja $c\in [a,b]$, da velja: $f'(c) = 0$
	$$g'(c) = f'(c) - \frac{f(b) - f(a)}{b - a} = 0$$
	$$f'(c) = \frac{f(b) - f(a)}{b - a}$$
\end{dokaz}
\begin{Posledica}
	Naj bo f zvezna na $[a,b]$ in odvedljiva na $(a,b)$. Denimo, da je $f'(c) = 0$ za vsak $c\in [a,b]$. Potem je $f$ konstanta.
\end{Posledica}
\begin{dokaz}
	Izberimo poljuben $x \in [a,b]$. Uporabimo Lagrangev izrek na intervalu $[a,x]$. Obstaja $c \in [a,x]$:
	\newline
	$ 0 = f'(c) = \frac{f(x) - f(a)}{x - a}$
	\newline
	Iz tega sledi, da je $f(a) = f(x)$.
	Torej sledi, da je $f(a) = f(x) \quad \forall x\in [a,b]$. Torej je f konstanta.
\end{dokaz}
\begin{Izrek}
	Naj bosta f in g zvezni na $[a,b]$ in odvedljivi na $(a,b)$. Naj velja: $f'(x) = g'(x)$ za vsak $x$ na $[a,b]$. Potem obstaja $c \in \mathbb{R}$ in velja: $f(x) = g(x) + c$ za vsak $x \in [a,b]$.
\end{Izrek}
\pagebreak
\begin{Trditev}
	Naj bo f zvezna na $[a,b]$ in odvedljiva na $(a,b)$. Potem velja:
	\begin{enumerate}
		\item
		f je \textbf{naraščajoča} na $[a,b) \iff f'(x) \ge 0$ za vse $x \in (a,b)$
		\item
		f je \textbf{padajoča}  na $[a,b] \iff f'(x) \le 0$ za vse $x\in (a,b)$.
		\item
		če je $f'(x) > 0$ za vse $x \in (a,b) \Rightarrow$ f je strogo naraščajoča na $[a,b]$
		\item
		če je $f'(x) < 0$ za vse $x \in (a,b) \Rightarrow$ f je strogo padajoča na $[a,b]$
	\end{enumerate}
\end{Trditev}
\begin{Definicija}
	Naj bo f odvedljiva funkcija na množici D. Rešitve enačb $f'(c) = 0$ imenujemo stacionarne točke.
\end{Definicija}
\begin{Izrek}
	Naj bo f zvezna na $[a,b]$ in odvedljiva na $(a,b)$. Naj bo $x_0$ stacionarna točka odvedljive funkcije f.
	\begin{enumerate}
		\item
		Če je v neki okolici točke $x_0$, levo od $x_0$ odvod nenegativen in desni od $x_0$ nepozitiven, potem ima f v točki $x_0$ lokalni maksimum.
		\item
		Če v neki okolici točke $x_0$ odvod ne spremeni predznaka, potem f v točki $x_0$ nima lokalnih ekstremov.
	\end{enumerate}
\end{Izrek}
\begin{Opomba}
	Zvezno odvedljivi funkciji se predznak odvoda spremeni samo v stacionarni točki.
\end{Opomba}
\begin{Izrek}
	Naj bo f dvakrat zvezno odvedljiva funkcija in $x_0$ njena stacionarna točka.
	\newline
	$f''(x_0) > 0 \Rightarrow$ f v točki  $x_0$  lokani minimum.
	\newline
	$f''(x_0) < 0 \Rightarrow $ f v točki $x_0$ lokalni maksimum.
\end{Izrek}
\begin{Izrek}
	Naj bo f zvezna funkcija na $[a,b]$ in odvedljiva na $(a,b)$. Potem funkcija f doseže globalni ekstrem v stacionarni točki ali v krajišču intervala $[a,b]$.
\end{Izrek}
\begin{Definicija}
	Naj bo f odvedljiva funkcija v okolici točke a. Če v vsaki točki iz te okolice velja, da je graf nad tangento na graf v točki $a$, potem je f \textbf{konveksna} v točki a.
	\newline
	Če v vsaki točki iz te okolice velja, da je graf pod tangetno na graf v točki a, potem je f \textbf{konkavna} v točki a.
\end{Definicija}
\begin{Izrek}
	Naj bo funkcija f na $(a,b)$ dvakrat odvedljiva.
	\begin{itemize}
		\item
		Če je $f''(x) > 0$ za vsak $x \in (a,b)$, potem je f konveksna na $(a,b)$.
		\item
		Če je $f''(x) < 0$ za vsak $x\in (a,b)$, potem je f konkavna na $(a,b)$.
	\end{itemize}
\end{Izrek}
\begin{dokaz}
	Izberimo $x_0 \in (a,b)$, dokazujemo, da je graf f pod tangento v točki $x_0$. Enačba tangente: $y -f(x_0) = f'(x_0)(x-x_0)$.
	\newline
	Po Lagrangu: $f(x) - f(x_0) = f'(c) (x - x_0)$. Iz $f''(x) > 0$ sledi, da je $f'$ na $(a,b)$ strogo naraščajoča funkcija. Če je $x > x_0$, je $c > x_0$ in od tod sledi : $f'(c) > f'(x_0)$.
	$$f(x) = f(x) + f'(c) (x -x_0) > f(x_0)+ f'(x_0)(x - x_0) = y$$
	Zato je graf nad tangento.(glej zvezek)
\end{dokaz}
\begin{Definicija}
	Naj bo f odvedljiva funkcija. Pravimo, da ima f v točki a \textbf{prevoj}, če velja bodisi, da je levo od a konveksna in desno od a konkavna, bodisi obratno.
\end{Definicija}
\begin{Opomba}
	Če je f v točki a dvakrat zvezno odvedljiva in ima v točki a prevoj, potem velja: $f''(a)= 0$. Obratno ni nujno res: če je $f''(a) = 0$, potem a ni nujno prevoj.
\end{Opomba}
\begin{Izrek}
	(L'Hospitalovo pravilo) Naj bosta u in v odeveljivi funkciji v okolici točke a. Denimo, da je $u(a) = v(a)$ in da v neki okolici točke a funkcija v in funkcija v' nimata ničel razen v a. Če obstajata:
	$\lim\limits_{x \to a}{\frac{u'(x)}{v'(x)}}$, potem obstaja $\lim\limits_{x \to a}{\frac{u(x)}{v(x)}}$ in velja:
	$$\lim\limits_{x \to a}{\frac{u'(x)}{v'(x)}} = \lim\limits_{x \to a}{\frac{u(x)}{v(x)}}$$
\end{Izrek}
\begin{Izrek}
	Naj bosta u in v odvedljivi funkciji v okolici točke a, razen morda v a in denimo, da je $\lim\limits_{x \to a}{u(x)} = \pm \infty$ in da je $\lim\limits_{x \to a}{v(x)} = \pm \infty$. Če obstaja končna in neskončna  $\lim\limits_{x \to a}{\frac{u'(x)}{v'(x)}}$, potem obstaja  $\lim\limits_{x \to a}{\frac{u(x)}{v(x)}}$ in velja
	$$ \lim\limits_{x \to a}{\frac{u(x)}{v(x)}} =  \lim\limits_{x \to a}{\frac{u'(x)}{v'(x)}}$$
\end{Izrek}
\section{Integral}
\subsection{Nedoločen integral}
\begin{Definicija}
	Naj bo funkcija f definirana na intervalu I. Odvedljiva funkcija F, definirana na intervalu I za katero velja: $F'(x) =f(x)$ za vse $x \in I$ imenujemo \textbf{primitivna funkcija} funkcije f.
\end{Definicija}
\begin{Opomba}
	Če je F primitivna funkcija od f na I, potem je tudi funkcija \newline
	$x \mapsto F(x) + C \quad C\in \mathbb{R}$, primitivna funkcija od f na I.
\end{Opomba}
\begin{Definicija}
	Naj bo funkcija f definirana na intervalu I. Družino vseh primitivnih funkcij funkcije f na I imenujemo \textbf{nedoločen integral} funkcije f in označimo
	$$\int{f(x)} dx$$
\end{Definicija}
\begin{Trditev}
	Denimo, da imata f in g primitivni funkciji. Potem imata tudi funkciji $f+g$ in $k\cdot f \quad k\in \mathbb{R}$ primitivna funkcija in velja:
	$$\int{(f(x) + g(x))}dx = \int{f(x) dx} + \int{g(x) dx}$$
	$$\int{k \cdot f(x) dx} = k \cdot \int{f(x) dx}$$
\end{Trditev}
\begin{dokaz}
	Denimo, da sta F in G primitivni funkciji od f in g.
	\newline
	Velja: $F' = f$ in $G' = g$
	\newline
	$(F+G)' = F' + G' = f + g$
	\newline
	\newline
	$k\in \mathbb{R} \quad (k\cdot F)' = k \cdot f$;
	\newline
	$\int{k \cdot f(x)}dx = k \cdot F(x) + C = k \int{f(x) dx}$
\end{dokaz}
\begin{Trditev}
	(Uvedba nove spremenljivke)
	Če ima funkcija f primitivno funkcijo in je $x = x(t)$ odvedljiva funkcija, potem je tudi funkcija $f(x(t)) \cdot \dot{x}(t)$ primitivna funkcija in velja:
	$$\int{f(x) dx} = \int{ f(x(t)) \cdot \dot{x}(t)}dt$$
\end{Trditev}
\begin{Trditev}
	(Integracija po delih - Per Partes) Naj bosta f in g odvedljivi funkciji. Če obstaja eden od integralov $\int{f(x) g'(x)}dx$ ali $\int{f'(x) g(x)}dx$, potem obstaja tudi drugi in velja:
	$$\int{f(x) g'(x)}dx + \int{f'(x) g(x)} =f(x) \cdot g(x)$$
	\\
	\\
	$$\int{u dv} = u \cdot v - \int{v du}$$
\end{Trditev}
\subsection{Določeni integral}
\begin{Definicija}
	Naj bo $[a,b]$ zaprti interval. Končno mnogo točk $x_0, x_1,\cdots , x_n$ imenujemo \textbf{delitev} intervala $[a,b]$, če velja:
	$$x_0 = a < x_1 < x_2 < \cdots < x_{n-1} < x_n = b$$
	Označimo: $D = \{x_0, x_1, \cdots x_n\}$
\end{Definicija}
\begin{Definicija}
	Naj bo f omejena funkcija na intervalu $[a,b]$. Velja:
	$$m_j = \inf \{f(x); x\in [x_{j-1},x_j]\}$$
	$$M_j = \sup \{f(x); x\in [x_{j-1},x_j]\}$$
\end{Definicija}
\begin{Definicija}
	$$S(D) =  \sum_{j= 0}^{n}{M_j(x_j - x_{j-1})}$$
	imenujemo \textbf{zgornja Darbouxova vsota}.
	
	$$s(D) = \sum_{j =1}^{n}{m_j(x_j -x_{j-1})}$$
	imenujemo \textbf{spodnja Darbouxova vsota}.
	\newline
	Velja: $s(D) \le S(D)$.
\end{Definicija}
\begin{Definicija}
	Naj bo $$s = \sup \{s(D); D \quad \textnormal{delitev} \quad [a,b]\}$$ in \newline $$S = \inf  \{S(D); D \quad \textnormal{delitev} \quad [a,b]\}.$$ Če je $ s = S$, potem pravimo, da je f \textbf{integrabilna} na $[a,b]$ in $S$ imenujemo \textbf{določeni integral} funkcije f na $[a,b]$.
	\newline
	Če $ S\ne s$, potem f ni integrabilna na $[a,b]$.
\end{Definicija}
\begin{Definicija}
	Naj bo $f:[a,b] \to \mathbb{R}$ funkcija in naj bo $D$ delitev $[a,b]$. Naj bo $T$ množica točk, ki je usklajena z delitvijo $D = \{x_0, x_1, \cdots x_n\}$ in $x_0 = a < x_1 < x_2 < \cdots < x_{n-1} < x_n = b$, potem je $$T_D = \{t_1, t_2, t_3,\cdots t_n\}$$ in $t_j \in [t_j,t_{j-1}]$. Množici $T_D$ pravimo \textbf{usklajena izbira testnih točk}.
\end{Definicija}
\begin{Definicija}
	Naj bo  $f:[a,b] \to \mathbb{R}$ funkcija, $D$ delitev $[a,b]$ in naj bo $T_D$ usklajena izbira testnih točk. Potem:
	$$R(f,D,T_D) = \sum_{j = 1}^{n}{f(t_j)(x_j - x_{j-1})}$$
	imenujemo \textbf{Riemmanova vsota}.
\end{Definicija}
\begin{Definicija}
	Naj bo  $f:[a,b] \to \mathbb{R}$ funkcija, $D$ delitev $[a,b]$ in naj bo $T_D$ usklajena izbira testnih točk. Potem:
	$$\int_{b}^{a}{f(x) dx} = \lim\limits_{\delta(D) \to 0}{(R(f,D,T))}$$ imenujemo \textbf{Riemmanov integral}, pri čemer je $\delta(D) = \max\{x_j - x_{j-1} , j = 1, \dots ,n\}$ velikost delitve D.
\end{Definicija}
\begin{Definicija}
	$\forall \varepsilon > 0, \exists \delta >0$, da velja: za vsako delitev D, za katero je $\delta(D) < \delta$ in za vsako usklajeno izbiro testnih točk $T_D$:
	$$|R(f,D,T_D) - I| < \varepsilon$$
\end{Definicija}
\begin{Opomba}
	Izkaže se, da omejena funkcija f je integrabilna $\iff$ integrabilna v Riemmanovem smislu.
\end{Opomba}
\begin{Opomba}
	Če funkcija $f:[a,b] \to \mathbb{R}$ ni omejena, potem ni Riemmanovo integrabilna.
\end{Opomba}
\begin{Trditev}
	Naj bo $f:[a,b] \to \mathbb{R}$ monotona funkcija. Potem je f integrabilna.
\end{Trditev}
\begin{Izrek}:
	\begin{itemize}
		\item
		Če je $f \ge 0$, potem je integral $\int_{a}^{b}{f(x) dx}$, ploščina lika pod grafom.
		\item
		$\int_{a}^{b}{f(x)dx} = \int_{a}^{b}{f(t) dt}$	
		\item
		$\int_{a}^{a}{f(x) dx} = 0$
		\item
		$\int_{a}^{b}{f(x) dx} = \int_{a}^{c}{f(x) dx} + \int_{c}^{b}{f(x) dx}$	
	\end{itemize}
\end{Izrek}
\begin{Opomba}
	Dogovor: funkcija $f:[a,b] \to \mathbb{R}$ integrabilna: $$\int_{a}^{b}{f(x) dx} = - \int_{b}^{a}{f(x) dx}$$
\end{Opomba}
\begin{Trditev}
	Naj bosta f in g integrabilni funkciji na $[a,b]$:
	\begin{itemize}
		\item
		$$\int_{a}^{b}{c\cdot f(x) dx} = c\cdot  \int_{a}^{b}{f(x) dx} \quad c\in \mathbb{R}$$
		\item
		$$\int_{a}^{b}{(f(x)+g(x)) dx} =\int_{a}^{b}{ f(x) dx} + \int_{a}^{b}{g(x) dx}$$
		\item
		$$\int_{a}^{b}{(f(x)- g(x)) dx} =\int_{a}^{b}{ f(x) dx} - \int_{a}^{b}{g(x) dx}$$
		\item
		Če velja $m\le f(x) \le M$ za vse $x\in [a,b]$, potem velja:
		$$(b-a) m \le \int_{a}^{b}{ f(x) dx} \le M(b-a)$$
		\item
		$$\arrowvert \int_{a}^{b}{f(x) dx}\arrowvert \le  \int_{a}^{b}{f(x) dx}$$
	\end{itemize}
\end{Trditev}
\begin{Izrek}
	(Izrek o povprečni vrednosti) 
	Naj bo f integrabilna funkcija na $[a,b]$.
	Naj bosta $m = \inf\{f(x); x\in[a,b]\}$ in 
	$M = \sup\{f(x); x\in[a,b]\}$.
	\\
	Potem obstaja število $p\in [m,M]$, da velja:
	$$p = \frac{1}{b - a}\int_{a}^{b}{f(x)dx}$$
	\\
	Če je f zvezna na $[a,b]$, potem obstaja $c\in [a,b]$, da velja: $f'(c) = p$.
	Število p imenujemo povprečna vrednost funkcije f na $[a,b]$.
\end{Izrek}
\begin{dokaz}
	 $m\le f(x) \le M$ za vse $x\in [a,b]$.
	 \newline
	 Vemo : $$(b-a) m \le \int_{a}^{b}{ f(x) dx} \le M(b-a)$$ delimo s $(b-a)$ in dobimo:
	 \newline
	 $$m \le p\le M$$
	 \newline
	 Zvezne funkcije dosežejo vse vrednosti med min in max, zato obstaja $c\in[a,b]: f'(c) = p$.
\end{dokaz}
\begin{Izrek}
	Naj bo $f:  [a,b]\to \mathbb{R}$ zvezna funkcija. Potem je f integrabilna na $[a,b]$.
\end{Izrek}
\begin{dokaz}
	(Ideja dokaza)
	$s(D) \ge s = S \ge S(D)$
	\newline
	Poiščemo delitev D:
	$S(D) - s(D) < \varepsilon$
	\newline
	$\sum_{j = 1}^{n}{(M_j - m_j)(x_j - x_{j-1})}$
	\newline
	$(M_j - m_j)$ zaradi zveznosti lahko naredimo poljubno majhen.
\end{dokaz}
\begin{Posledica}
	Elementarne funkcije, ki so definirane na zaprtem intervalu $[a,b]$ so integrabilne na $[a,b]$.
\end{Posledica}
\begin{Definicija}
	Naj bo $f:[a,b]\to \mathbb{R}$ integrabilna funkcija. Definirajmo predpis: $F:[a,b]\to \mathbb{R}$:
	$$F(x) = \int_{a}^{x}{f(t)dt} \quad x\in[a,b]$$
	Funkcijo F imenujemo integral kot funkcijo zgornje meje.
\end{Definicija}
\begin{Izrek}
	Naj bosta f in F definirana kot zgornja definicija. Potem velja:
	\begin{itemize}
		\item 
		F je zvezna funkcija
		\item 
		Če je f zvezna funkcija, potem je F odvedljiva in velja:
		$$F'(x) =f(x) \quad \forall x\in(a,b)$$ Torej je F primitivna funkcija od f.
	\end{itemize}
\end{Izrek}
\begin{Izrek}
	Če je f zvezna funkcija na $[a,b]$ in F njena primitivna funkcija, potem $\int_{a}^{b}{f(x)dx} = F(b) - F(a)$
\end{Izrek}
\begin{dokaz}
	Naj bo F primitivna funkcija f. Vemo, da je $x\mapsto \int_{0}^{x}{f(t) dt}$ tudi primitivna funkcija od f. Ker se primitivni funkciji na intervalu $[a,b]$ lahko razlikujeta kvečjemu za konstanto je :
	\newline
	$F(x) = \int_{a}^{x}{f(t)dt} + C \quad c\in\mathbb{R}$
	\newline
	$F(a) = C$
	\newline
	$F(b) = \int_{a}^{b}{f(t) dt} + C$
	
\end{dokaz}
\begin{Izrek}
	Naj bo funkcija f zvezna na $[a,b]$ in naj bo $g: [\alpha, \beta]\to [a,b]$ zvezno odvedljiva funkcija. Potem velja:
	$$\int_{g(\alpha)}^{g(\beta)}{f(x) dx} = \int_{\alpha}^{\beta}{f(g(t)) \cdot g'(t) dt}$$
\end{Izrek}
\begin{dokaz}
	Ker je funkcija f zvezna, potem obstaja funkcija F za katero velja:
	\\
	$F'(x) =f(x) \quad \forall x\in(a,b)$
	\\
	$\int_{g(\alpha)}^{g(\beta)}{f(x) dx} = F(x) |^{g(\beta)}_{g(\alpha)} = F(g(\beta)) - F(g(\alpha))$
	\newline
	Odvajamo $F\circ g$: $(F\circ g)' = F'g \cdot g' = f(g) - g'$
	\newline
	$F\circ g$ je primitivna funkcija od $(f\circ g)\cdot g' $, zato po Newton - Leibnizovi formuli:
	$$\int_{\alpha}^{\beta}{f(g(t)) g'(t) dt} = F\circ g \|^\beta _\alpha = F(g(\beta)) - F(g(\alpha))$$
\end{dokaz}
\begin{Izrek}
	Naj bosta u in v odvedljivi funkciji na $[a,b]$. Potem velja:
	$$\int_{a}^{b}{u(x)\cdot v'(x) dx} = (u(x) -v(x))| - \int_{a}^{b}{u'(x) \cdot v(x) dx}$$
\end{Izrek}
\begin{dokaz}
	$(u\cdot v)' = u' \cdot v + u \cdot v'$
	\newline
	$u\cdot v$ je primitivna funkcija od $u'v + v'u$
	\newline
	Zato po Newton - Leibniz formuli velja:
	$$\int_{a}^{b}{(u'(x)v(x) + u(x)v'(x) ) dx }= (u(x) v(x))|$$
\end{dokaz}
\subsection{Posplošeni integral}
\begin{Definicija}
	Naj bo funkcija $f:[a,b) \to \mathbb{R}$ zvezna funkcija. Potem je \textbf{posplošen integral} funkcije f v mejah od a do b.
	$$\int_{a}^{b}{f(x)\ dx} = \lim\limits_{c\to b}\int_{a}^{c}{f(x)\ dx}$$
	če ta limita obstaja.
	\\
	V tem primeru pravimo, da je f posplošeno integrabilna na intervalu $[a,b)$, ali da integral konvergira. Sicer pravimo, da integral divergira.
\end{Definicija}
\begin{Definicija}
	Naj bo $f:(a,b] \to \mathbb{R}$ zvezna funkcija. Potem je $f$ posplošeno itegrabilna na intervalu $(a,b]$, če obstaja:
	$$\lim\limits_{c\searrow a}{\int_{c}^{b}{f(x) \ dx}}$$
	V tem primeru jo označimo $\int_{a}^{b}{f(x)\ dx}$.
\end{Definicija}
\begin{Definicija}
	Naj bo $a < c < b$ in naj bo $f:[a,c) \cup (c,d] \to \mathbb{R}$ zvezna funkcija. Pravimo, da je f posplošeno integrabilna na intervalu $[a,b]$, če je f posplošeno integrabilna na intervalu $a,c$ in $c,b$ in definiramo:
	$$\int_{a}^{b}{f(x) \ dx} = \int_{a}^{c}{f(x) \ dx} + \int_{c}^{b}{f(x) \ dx} = \lim\limits_{t \nearrow c}{\int_{a}^{t}{f(x) \ dx}} + \lim\limits_{b\searrow c}{\int_{b}^{nevem}{f(x) \ dx}}$$
\end{Definicija}
\begin{Opomba}
	\begin{itemize}
		\item
		Podobno nadaljujemo v bolj kompliciranih primerih: da bi integral obstajal, morajo obstajati vsi integrali na manjših intervalih.
		\item
		Ta definicija posplošene integrabilnosti ni isto kot:
		$$\int_{-1}^{1}{\frac{1}{x} \ dx} = \lim\limits_{t\searrow 0}{\left( \int_{-1}^{-t}{\frac{1}{x} \ dx} + \int_{t}^{1}{\frac{1}{x}\ dx}\right)} = \lim\limits_{t 
		\searrow 0}{(0)} = 0$$
		$$\lim\limits_{t \searrow 0}{\int_{t}^{1}{\frac{1}{x} \ dx}} = \lim\limits_{t\searrow 0}{(\ln{x}|^1_0)} = \lim\limits_{t \searrow 0}{(-\ln{t})}= + \infty$$
		Posplošeni integral divergira, glavna vrednost pa ostaja.
	\end{itemize}
\end{Opomba}
\begin{Izrek}
	Naj bo g zvezna funkcija na $[a,b]$. Potem:
	$$\int_{a}^{b}{\frac{g(x)}{(x-a)^s}\ dx} $$
	konvergira, če $s < 1$, in divergira, če je $s \le 1$ in je $g(a) \ne 0$.
\end{Izrek}
\begin{Definicija}
	Naj bo f definirana $f:[a,\infty) \to \mathbb{R}$, ter zvezna funkcija. Če obstaja 
	$$\lim\limits_{M\to \infty}\int_{a}^{M}{f(x)dx},$$ potem pravimo, da je f posplošeno integrabilna na $[a,\infty)$ in limito označimo z $$\int_{a}^{\infty}{f(x) dx}$$.
\end{Definicija}
\begin{Opomba}
	Definiciji posplošene integrabilnosti na končnem in neskončnem intervalu združimo v strogem smislu:
	Če je $f: [a,\infty)\backslash \{c_1, c_2,\dots c_n\} \to \mathbb{R}$ zvezna funkcija potem je f posplošen integrabilna, če obstajata posplošena integrala $\int_{a}^{M}{f(x)dx}$ in $\int_{M}^{\infty}$, kjer je $M> \max\{c_1,c_2, \dots c_n\}$.
\end{Opomba}
\begin{Izrek}
	Naj bo $g: [a,\infty) \to \mathbb{R}$ zvezna in omejena funkcija, $a>0$. Potem integral $$\int_{a}^{\infty}{\frac{g(x)}{x^{s}}dx}$$ obstaja, če je $s > 1$ in  ne obstaja, če je $s \le 1$ in je g v neskončnosti omejena stran od 0. (tj.$\exists n>0$, da je $g(x) >n$, za vse dovolj velike $x$ ali pa $g(x) \le -m$ za vse dovolj velike x).
\end{Izrek}
\begin{Opomba}
	Pogoj v neskončnost je gotovo izpolnjen v primeru, če obstaja: $\lim\limits_{x\to \infty}{g(x)}\ne 0$.
\end{Opomba}
\begin{Trditev}
	Denimo, da sta f in g definirani na intervalu: $[a,\infty) \to \mathbb{R}$ zvezni funkciji in denimo, da velja $0\le f(x) \le g(x)$ za vse $x\in [a,\infty)$:
	\begin{enumerate}
		\item
		Če obstaja $\int_{a}^{\infty}{g(x) dx}$, potem obstaja $\int_{a}^{\infty}{f(x) dx}$.
		\item
		Če divergira $\int_{a}^{\infty}{f(x) dx}$, potem divergira tudi $\int_{a}^{\infty}{g(x) dx}$
	\end{enumerate}
\end{Trditev}
\begin{Izrek}
	(Integralski kriterij za konvergenco vrst)
	Naj bo $\sum_{n=1}^{\infty}{a_n}$ vrsta s pozitivnimi členi, za katero obstaja funkcija $f: [a,\infty) \to \mathbb{R}$, ki je pozitivno zvezna, padajoča in $f(n) = a_n$. Tedaj vrsta $\sum_{n=1}^{\infty}{a_n}$ konvergira natanko tedaj, kadar obstaja:
	$$\int_{1}^{\infty}{f(x) dx}$$
\end{Izrek}
\subsection{Numerično integriranje}
Včasih določenega integrala ne moremo izračnati natančno, radi pa bi ocenili njegovo približno vrednost.
$$\int_{a}^{b}{f(x) \ dx} = \int_{a}^{b}{g(x) \ dx} + R$$
Funkcijo $f$ nadomestimo s prbiližkom $g$, pri integriranju naredimo napako $R$.
\subsubsection{Trapezna metoda}
Interval $[a,b]$ razdelimo na $n$ enakih delov: $x_k = a + k \frac{b-a}{n}$ in naj bo
$g$ odsekovo linearna funkcija skozi točke $(x_k, f(x_k))$. 
$$\int_{x_{k-1}}^{x_k}{g(x) \ dx} = \frac{f(x_{k-1}) + f(x_k)}{2} \cdot \frac{b-a}{n}$$
$$\int_{a}^{b}{g(x) \ dx} = \frac{b-a}{2n} (f(x_0) + f(x_1) +f(x_1)+ f(x_2) + \dots + f(x_{n-2}) + f(x_{n-1})+f(x_{n-1}) + f(x_n)) $$
$$= \frac{b-a}{2n}(f(x_0) + 2f(x_1)+ 2f(x_2)+ \dots + 2f(x_{n-2}) + 2f(x_{n-1}) + f(x_n))$$
Formula za napako:
$$|R_n| \ge \frac{(b-a)^3}{12 n^2} \max_{x\in [0,1]}|f''(x)|$$
\section{Taylorjeva vrsta}
\begin{Definicija}
	Naj bo f vsaj $n-$ krat zvezno odvedljiva v okolici točke $a$:
	$$T_n(x) = f(a) + f'(a) (x-a) + \frac{f''(a)}{2}(x-a)^n + \frac{f^{(3)}(a)}{3!}(x-a)^3 + \dots + \frac{f^{(n)}(a)}{n!}(x-a)^n$$
	$T_n(x)$ imenujemo n-ti \textbf{Taylorjev polinom} funkcije f v točki a.
\end{Definicija}
\begin{Trditev}
	V okolici točke a lahko zapišemo: 
	$f(x) = T_n(x) + R_n(x)$. S polinomom $T_n(x)$ aproksimiramo funkcijo f(x) v okolici točke a. Velikost $R_n(x)$ pove kako dobra je aproksimacija.
	$$R_n(x) = \frac{f^{(n+1)(c)}}{(n+1)!} (x-a)^{n+1} \quad c\in[a,x]$$
\end{Trditev}
\begin{Definicija}
	Denimo, da je f neskončnokrat odvedljiva v okolici točke a. Potem lahko funkciji f priredimo vrsto:
	$$T(x) = \sum_{n = 0}^{\infty}{\frac{f^{(n)}(a)}{n!}(x-a)^n}$$
	$T(x)$ imenujemo \textbf{Taylorjeva vrsta} prirejena funkciji f v točki a.
	\\
	$T_n(x)$ je n-ta delna vsota vrste $T(x)$.
\end{Definicija}
\begin{Opomba}
	Vrsta $T(x)$ konvergira natanko tedaj, kadar konvergira njeno zaporedje delnih vsot, tj. $T_n(x)$, tj. natanko tedaj ko zaporedje $R_n(x)$ konvergira proti 0.
\end{Opomba}
\begin{Opomba}
	Če vemo samo to, da je $T(x)$ konvergentna ne vemo pa, da gre $R_n(x)$ proti 0, potem $f(x) = T(x) $ ni nujno res.
\end{Opomba}
\begin{Zgled}
	Eksponentna funkcija v točki 0:
	$$e^x= 1 + x +\frac{x^2}{2} + \frac{x^3}{3!} + \cdot + \frac{x^n}{n!} = \sum_{n= 0}^{\infty}{\frac{x^n}{n}} \quad \forall x\in \mathbb{R}$$
	\\
	Posledica: $$\sum_{n= 0}^{\infty}{\frac{1}{n!}} = e$$
\end{Zgled}
\begin{Zgled}
	Funkcija sinus in kosinus v točki 0:
	$$\sin(x) = x - \frac{x^3}{3!} + \frac{x^5}{5!} - \frac{x^7}{7!} + \dots = \sum_{n= 0}^{ \infty}{\frac{(-1)^n \ x^{2n+1}}{(2n+1)!}} \quad \forall x\in \mathbb{R}$$
	
	$$\cos(x) = 1- \frac{x^2}{2} + \frac{x^4}{4!} - \frac{x^6}{6!} + \dots = \sum_{n =0}^{\infty}{\frac{(-1)^n \ x^{2n}}{2n!}} \quad \forall x\in \mathbb{R}$$
	
\end{Zgled}
\begin{Zgled}
	Binomska vrsta:
	$$(1+x)^\alpha = \sum_{n = 0}^{\infty}{\binom{\alpha}{n}x^n} \quad x\in(-1,1), \alpha\in\mathbb{R}$$
	
\end{Zgled}
\begin{Zgled}
	Logaritemska vrsta v okolici točke 0:
	$$\ln(x+1) = x- \frac{x^2}{2} + \frac{x^3}{3} - \frac{x^4}{4} + \dots  = \sum_{n= 1}^{\infty}{\frac{(-1)^{n+1}}{n}x^n} \quad  |x|< 1$$
\end{Zgled}
\section{Funkcijska zaporedja in vrste}
\begin{Definicija}
	Naj bo $I\subset \mathbb{R}$ interval in naj bodo $f: I \to \mathbb{R}$ funkcije za vsak $j \in \mathbb{N}$. Potem $$\{f_j: I\to \mathbb{R}\}_{j\in \mathbb{N}}$$ imenujemo \textbf{funkcijsko zaporedje}.
\end{Definicija}
\begin{Opomba}
	Taylorjeva vrsta je funkcijska vrsta.
\end{Opomba}
\begin{Definicija}
	Naj bo $I \subset \mathbb{R}$ interval in naj bo $\{f_n: I\to \mathbb{R}\}_{n\in \mathbb{N}}$ funkcijsko zaporedje. Pravimo, da $\{f_n\}$ \textbf{konvergira po točkah} na I, če za vsak $x\in I $ konvergira številsko zaporedje $\{f_n(x)\}_{n\in \mathbb{N}}$. V tem primeru definiramo: $$f(x) = \lim\limits_{x\to \infty}{f_n(x)} \quad \forall x\in I$$
	in funkcijo $f:I\to \mathbb{R}$ imenujemo\textbf{ limitna funkcija}.
\end{Definicija}
\begin{Definicija}
	Naj bo $I \subset \mathbb{R}$ interval in $\{f_n: I\to \mathbb{R}\}_{n\in \mathbb{N}}$ funkcijsko zaporedje. Pravimo, da funkcijsko zaporedje $\{f_n: I\to \mathbb{R}\}_{n\in \mathbb{N}}$ \textbf{enakomerno konvergira} proti funkciji $f: I \to \mathbb{R}$ na I, če za vsak $\varepsilon > 0$ , obstaja $n_0\in \mathbb{N}$, da velja $|f_n(x) -f(x)| <\varepsilon $ za vse $x\in I$ in za vse $n\ge n_0$
\end{Definicija}
\begin{Opomba}
	Funkcijsko zaporedje $\{f_n\}$ konvergira po točkah na $I$ proti$f$, če velja:
	$$\forall x\in I \quad \varepsilon > 0, \ \exists n_0\in\mathbb{N}$$, da velja $|f_n(x) -f(x)| < \varepsilon$ za vsak $n\ge n_0$.	
\end{Opomba}
\begin{Opomba}
	Če zaporedje $\{f_n : I\to \mathbb{R} \}$ enakomerno konvergira proti $f: I\to \mathbb{R}$ na I, potem zaporedje $\{f_n : I\to \mathbb{R} \}$ konvergira proti $f: I \to \mathbb{R}$ po točkah na I. Obratno ni nujno res!
\end{Opomba}
\begin{Definicija}
	Naj bo $\{u_n : I \to \mathbb{R} \}$ funkcijsko zaporedje na $I\subset \mathbb{R}$. Funkcijska vrsta je $$\sum_{n = 1}^{\infty}{u_n(x)}$$
	Pravimo, da funkcijska vrsta konvergira po točkah  na $I$, če za vsak $x\in I$ konvergira številska vrsta $\int_{n=1}^{\infty}{u_n(x)}$. Pravimo, da funkcijska vrsta konvergira enakomerno na $I$, če zaporedje delnih vsot konvergira enakomerno na I.
\end{Definicija}
\begin{Posledica}
	Naj bo  $\{u_n : I \to \mathbb{R} \}$ funkcijsko zaporedje zveznih funkcij na $I \subset \mathbb{R}$. Če vrsta $\sum_{n=1}^{\infty}{u_n(x)} $ konvergira enakomerno na $I$, potem je njena vsota zvezna funkcija.
\end{Posledica}
\begin{Izrek}
	(Weierstrassov M-test)
	Naj bo  $\{u_n : I \to \mathbb{R} \}$ funkcijsko zaporedje $I\subseteq \mathbb{R}$. Denimo, da obstaja zaporedje $\{c_n\}$ pozitivnih števil, da velja :
	$$|u_n(x)| \le c_n \quad \forall n\in \mathbb{N}, \forall x\in I$$
	Če je številska vrsta $\sum_{n=1}^{\infty}{c_n}$ konvergentna, potem je funkcijska vrsta $\sum_{n=1}^{\infty}{u_n(x)}$ enakomerno konvergentna na I (in absolutno konvergentna po točkah v I).
\end{Izrek}
\subsection{Potenčne vrste}
\begin{Definicija}
	Potenčne vrste so funkcijske vrste oblike
	$$\sum_{n=1}^{\infty}{c_n(x-a)^n} \quad a\in\mathbb{R},\ c_n\in\mathbb{R},\ \forall n$$
\end{Definicija}
\begin{Opomba}
	Če je $a =0:\quad \sum_{n= 1}^{\infty}{c_n\ x^n}$. 
\end{Opomba}
\begin{Opomba}
	Taylorjeve vrste so potenčne vrste.
\end{Opomba}
\begin{Izrek}
	Naj bo $\sum_{n=1}^{\infty}{c_n(x-a)^n}$ potenčna vrsta. Obstaja $R\in [a,\infty) \cup \infty$ z naslednjo lastnostjo:
	\begin{enumerate}
		\item
		če je x, $|x-a| < R$, potem  $\sum_{n=1}^{\infty}{c_n(x-a)^n}$ konvergira in absolutno konvergira.
		\item
		če je x, $|x-a| < R$, potem  $\sum_{n=1}^{\infty}{c_n(x-a)^n}$ divergira.
	\end{enumerate}
	Če je $r\in (0, R)$, potem  $\sum_{n=1}^{\infty}{c_n(x-a)^n}$ na  $[a-r, a+r]$ konvergira enakomerno.
	R imenujemo \textbf{konvergenčni polmer} potenčne vrste.
\end{Izrek}
\begin{Posledica}
	Vsota potenčne vrste $\sum_{n=1}^{\infty}{c_n(x-a)^n}$ s polmerom $R > 0$ je zvezna funkcija na $(a-R, a+R)$.
\end{Posledica}
\begin{Izrek}
	Dana je potenčna vrsta $\sum_{n=1}^{\infty}{c_n(x-a)^n}$. Za konvergenčni polmer velja:
	\begin{itemize}
		\item
		$$\frac{1}{R} = \lim\limits_{n = \infty}{\frac{|c_{n+1}|}{|c_n|}}$$
		če ta limita obstaja.
		\item
		$$\frac{1}{R} = \lim\limits_{n = \infty}{\sqrt[n]{c_n}}$$
		če ta limita obstaja.
	\end{itemize}
\end{Izrek}
\begin{Izrek}
	Naj bo $\sum_{n=1}^{\infty}{c_n(x-a)^n}$ potenčna vrsta. Za konvergenčni polmer R velja:
	$$\frac{1}{R} = \limsup{\sqrt[n]{|c_n|}}$$
\end{Izrek}
\begin{Opomba}
	Prednost: Ta formula vedno velja, ker $\limsup_{n\to \infty}$ vedno obstaja. (Dopuščamo vrednost $\infty$).
	\\
	Slabost: Limite ne znamo izračunati.
\end{Opomba}
\subsection{Integriranje in odvajanje funkcijskih zaporedij in vrst}
\begin{Izrek}
	Naj bo $\{f_n :[a,b]\to \mathbb{R} \}$ funkcijsko zaporedje zveznih funkcij. Predpostavimo, da $\{f_n\}$ enakomerno na $[a.b]$ konvergira proti $f:[a,b] \to \mathbb{R}$. Potem velja:
	$$\int_{a}^{b}{f(x)\ dx} = \int_{a}^{b}{(\lim\limits_{n\to \infty}{f_n(x)}) \ dx} = \lim\limits_{n\to \infty}{\int_{a}^{b}{f_n(x)}\ dx} $$
\end{Izrek}
\begin{Posledica}
	Naj bo $\{u_n: [a,b] \to \mathbb{R}\}$ funkcijsko zaporedje zveznih funkcij in denimo, da $\sum_{n=0}^{\infty}{u_n(x)}$ konvergira enakomerno na $[a,b]$.
\end{Posledica}
\begin{Izrek}
	Naj bo $\{f_n: [a,b] \to \mathbb{R}\}$ funkcijsko zaporedje zvezno odvedljivih funkcij na $[a,b]$. Denimo, da $f_n'$ konvergira enakomerno na $[a,b]$ proti $g$ in denimo, da za nek $c\in [a,b]$ številsko zaporedje $\{f_n(c) \} $ konvergira. Potem velja:
	\\
	Funkcijsko zaporedje $f_n$ enakomerno konvergira proti neki zvezno odvedljivi funkciji $f$ na $[a,b]$ in velja:
	$$f'(x) = \lim\limits_{n\to \infty}{f_n'(x)}$$
\end{Izrek}
\begin{Posledica}
	Naj bo $\{u_n: [a,b] \to \mathbb{R}\}$ funkcijsko zaporedje zvezno odvedljivih funkcij na $[a,b]$.
	Denimo, da $\sum_{n=0}^{\infty}{u_n'}$ konvergira enakomerno na $[a,b]$ in da vrsta $\sum_{n=0}^{\infty}{u_n(c)}$ konvergira za nek $c\in [a,b]$. Potem $\sum_{n=0}^{\infty}{u_n}$ enakomerno konvergira na $[a,b]$, njena vsota je zvezno odvedljiva na $[a,b]$ in velja:
	$$(\sum_{n=0}^{\infty}{u_n(x)})' = \sum_{n=0}^{\infty}{u_n'(x)}$$
\end{Posledica}
\begin{Izrek}
	(Odvajanje in integriranje potenčnih vrst)
	\\ 
	Naj bo $\sum_{n=0}^{\infty}{a_n \ x^n}$ potenčna vrsta s konvergenčnim polmerom R. Potem imata vrsti, ki ju dobimo s členskim odvajanjem $\sum_{n=0}^{\infty}{a_n \ n\ x^{n-1}}$ in s členskim integriranjem $\sum_{n=0}^{\infty}{\frac{a_n}{n+1} \ x^{n+1}}$, tudi konvergenčni polmer R in velja:
	\\
	$$f(x) = \sum_{n=0}^{\infty}{a_n \ x^n} \ \textrm{potem} \ f'(x) = (\sum_{n=0}^{\infty}{a_n \ x^n})' = \sum_{n=0}^{\infty}{a_n \ n\ x^{n+1}}$$
	
	$$\int_{0}^{x}{f(t) \ dt} = \int_{0}^{x}{(\sum_{n=0}^{\infty}{a_n \ x^n}) \ dt} = \sum_{n=0}^{\infty}{ \int_{0}^{x}{a_n \ x^n \ dt}}= \sum_{n=0}^{\infty}{a_n \ \frac{x^{n+1}}{n+1}}$$
\end{Izrek}
\begin{Posledica}
	Naj bo  $\sum_{n=0}^{\infty}{a_n \ x^n}$ potenčna vrsta s konvergenčnim polmerom R. Tedaj je njena vsota neskončnokrat odvedljiva na $(-R, R)$.
\end{Posledica}
\end{document}

